% Demo and instructions for the front end to the Qobitree tree-drawing package
% (LaTeX)

% To-consider (and do) for the documentation:
% - expand tree-dvips discussion?
% - incorporate how-to-draw picture?
% - mention hack for branches below qroof?

\documentclass[11pt]{article}

\usepackage{qtree}

\advance\textheight by 0.7in
\topmargin=-0.3in

\begin{document}

\begin{center}
{\bf Documentation for qtree, a \LaTeX\ tree package}\footnote{
Thanks to Jeff Siskind for permission to distribute the QobiTree code.
Please direct comments to Alexis Dimitriadis, {\em
alexis@babel.ling.upenn.edu}.}
\\ by {\em Jeffrey Mark Siskind,} \\ with
a front 
end by {\em Alexis Dimitriadis} \\[0.75ex]
~Version \qTreeVersion\\[1.5ex]
\end{center}

The {\it qtree\/} package consists of QobiTree, a package of tree-drawing
macros written by Jeff Siskind, and a front end that allows trees to be
specified in bracket notation, using whitespace to separate tokens.  Tree
nodes, which can have labels of any size or complexity, are automatically
arranged on the page, usually with quite good results.  Provisions exist for
fine-tuning the default layout.  The front end also centers trees (by default)
and provides some other nice features. 

A simple tree may look like this,
\begin{verbatim}
\Tree [.S This [.VP [.V is ] \qroof{a simple tree}.NP ] ]
\end{verbatim}
which produces:
\par
\Tree [.S This [.VP [.V is ] \qroof{a simple tree}.NP ] ]
\bigskip

The node labels in trees may be quite complicated; they may contain font
changes and math-mode text, line breaks introduced with \verb|\\| (which
produce centered lines), etc.  The trees produced have a maximum depth of 20,
with a maximum of five branches at any one node.  Unlike many other tree
macros, \emph{qtree} automatically adjusts for the width and height of tree
labels, and is pretty good at arranging nodes on the page. 

Trees are defined using a version of the bracket notation familiar to
linguists.  Tree elements are delimited by white space; braces can be
used to create multi-word labels.  \emph{Qtree} does not rely on
\verb|\catcode| changes for its operation, allowing trees to be
included in footnotes and other moving environments without problems.  

\section{Invocation}

{\tt Qtree.sty} is a \LaTeX\ package designed to be installed in a
directory of style files, and included with the \LaTeXe\ command 
\verb|\usepackage{qtree}|.

\paragraph{Postscript specials} \emph{Qtree} relies on \LaTeX's {\tt picture}
environment to draw the trees. 
Because this environment is rather limited, the lines used to draw trees look
better if \emph{qtree} is used with the package {\tt eepic.sty,} which
provides enhancements to the {\tt picture} environment.  This version of {\it
qtree\/} will automatically include {\tt eepic.sty\/} if it can find it, but
automatic inclusion can be suppressed by use of the package option {\tt
[noeepic].}  This may be necessary if the file will be processed with a driver
that does not understand PostScript specials (e.g., \emph{pdflatex}).  In that
case \emph{qtree} will be included as follows:

\begin{verbatim}
\usepackage[noeepic]{qtree}
\end{verbatim}

 
\paragraph{Tree centering} Trees are centered by default, but you can turn
centering off with  
the command \verb|\qtreecenterfalse|.  Normally this would be invoked in the
preamble, but it is possible to turn tree centering off and on (with the
corresponding \verb|\qtreecentertrue|) at any point.  These commands obey
normal scoping rules.  If used
inside, say, an example environment, their effect will only apply until the
end of that environment.  There is no provision for automatically
right-adjusted trees. 

\section{How to convert a tree to brackets}

Reading or writing a complex tree in bracket notation is not terribly easy for
humans; it helps to have an editor that can show matching braces as they are
typed in.  The procedure described here should allow you to easily convert a
tree to bracket notation.  If you don't have any difficulty with this, just
skip this section and do it any way you want!

\begin{enumerate}
\item Draw the tree you want to enter on a piece of paper, so you can look at
it. 

\item Imagine that the tree is a large peninsula, and your pencil is a boat
sailing around it.  Starting just to the left of the root node, move
downwards, following the outline of the tree until you come back to the root
node (on the right side, having moved counterclockwise around the tree),
without crossing any of the tree's lines.

\item \begin{enumerate}
\item
Every time you are at the left side of a non-terminal node, type a left
bracket, and the label for that node.
\item Every time you are
at a leaf node, type in the contents of that node.
\item
Finally, every time you are on the right of a non-terminal node, type a right
bracket (and the node name again, if you want to help keep them straight).
\end{enumerate}

\end{enumerate}

It's difficult to show all this without including a picture, but
consider the following tree; in the variant on the right, the numbered
subscripts show the order in which the brackets and labels are written.

\bigskip
\Tree [.A [.B [.C one ] [.D two ] ].B [.E three ] ].A
%
\Tree [.{[_1 A ]_{13}} 
  [.{[_2 B ]_9} [.{[_3 C ]_5} one_4 ] [.{[_6 D ]_8} two_7 ] ] 
	        [.{[_{10} E ]_{12}} three_{11} ] ]
\par
\medskip
Accordingly, we would generate the tree by typing the following:

{\center
\verb|\Tree [.A [.B [.C one ] [.D two ] ].B [.E three ] ].A| \par}

\section{Usage and features}

\paragraph{Syntax} The \emph{qtree} front end reads a tree description
written in the familiar (to linguists) bracket notation.  Tree labels are
delimited by whitespace.  To make a multi-word node label, enclose it in
braces; note also that 
\TeX\ discards the spaces immediately after {\it control
sequences\/} (commands whose name consists of a backslash followed by
letters), hence if a node label ends with a control sequence, like
\verb|\ldots| in the following example, you need to enclose it in braces
too.

\Tree [.CP [.{\sc Spec}(CP) {which car} ] {\ldots} ]
\begin{verbatim}
\Tree [.CP [.{\sc Spec}(CP) {which car} ] {\ldots} ]
\end{verbatim}

\paragraph{Label matching}
For convenience, a label for a non-terminal node can be written either after
the left bracket or after the right bracket corresponding to that node.  Thus
the following are equivalent:
\begin{verbatim}
\Tree [.S when [.NP the cat ]    sleeps ]
\Tree [.S when [    the cat ].NP sleeps ]
\end{verbatim}
To help keep braces matched when editing large trees, the front end allows
the option of writing a label after both the left and the right bracket of
the same node, as shown for the node NP below.  In this case the two labels
provided must be identical, token for token.
\begin{verbatim}
\Tree [.S when [.NP the cat ].NP sleeps ]
\end{verbatim}

\paragraph{No labels}

Sometimes we want to draw an abbreviated tree without a label at every
intermediate node.
\emph{Qtree} now draws such trees
properly, as in the following example.

\medskip
\Tree [.CP Spec(CP) [ C^0 Comp(CP) ] ]
\begin{verbatim}
\Tree [.CP Spec(CP) [ C^0 Comp(CP) ] ]
\end{verbatim}

\paragraph{Roofs}
It is possible to draw a triangular ``roof'' above a phrase that is treated as
a unit.  (See example on page \pageref{roof}). This is done with the command
\verb|\qroof|, which can appear anywhere a {\it leaf\/} can appear.  The slope
of the roof is equal to the ratio \verb|\qroofy / \qroofx| (these counters may
be reset to any pair of integers between zero and six; the default is 1/3).

To create a roof
labeled {\it NP\/} over the phrase {\it the book,} write
\begin{verbatim}
	\qroof{the book}.NP
\end{verbatim}
If the phrase contains line breaks introduced with \verb|\\|, the resulting
lines are flush left, not centered.  Again, it is possible for the ``phrase''
to be a construction of arbitrary complexity; but the roof is implemented as a
leaf node (it is not part of the original QobiTree), and so the syntax of
\verb|\qroof| does not allow further branches of the tree to appear {\it
under\/} the roof. 
% Although the parsing messes up, it IS possible to work qroof into a
% non-terminal label:
% \Tree [.A this [.{\qroof{Cheat}.ME } x [ below ] ]]


\paragraph{Subscripts, superscripts and primes}
Trees are constructed in a special environment in which things like
\verb|NP_i|, \verb|N^0|, automatically format their subscripts or
superscripts in math mode, giving NP$_i$ and N$^0$, respectively.  The
command that arranges this is called \verb|\automath|, and can be enabled
outside the tree environment, if desired. (It is turned off with
\verb|\noautomath|).  This feature relies on
\verb|\catcode| changes for its operation;
in trees that appear in footnotes or floats, all sub- or superscripts must be
explicitly placed in math mode, as you would ordinarily do.

As a further convenience, constructions like \verb|X$'$|, producing X\1, can
be abbreviated \verb|X\1|.  (If you simply type \verb|X'| you get X', with an
apostrophe rather than a prime).  There is also \verb|X\2|, producing X\2, and
\verb|X\0|, producing X\0.  These commands also arrange for subtle
improvements in the centering of labels that use them.

Here is an example using some of these features:

\label{roof}
\Tree 
[.IP [ Roses ].NP_i [.I\1 [ are ].I\0 
   [.VP t_i [ [ going ].V\0 \qroof{out of style}.PP ].V\1 ].VP 
].I\1 ]
\begin{verbatim}
\Tree 
[.IP [ Roses ].NP_i [.I\1 [ are ].I\0 
   [.VP t_i [ [ going ].V\0 \qroof{out of style}.PP ].V\1 ].VP 
].I\1 ]
\end{verbatim}
Granted, by the time the examples get this big, the bracketed format isn't
all that readable, but it's certainly no worse than any other tree format,
and you can add white space to make it a little better.


\section{Tree placement}

\paragraph*{Numbered examples etc.}
A tree generated with \emph{qtree} can be placed in a numbered example
environment, in 
\verb|\parbox|es, inside math formulas, tables, pictures, etc.  The tree nodes
can also contain arbitrarily complex material---although, unfortunately, it is
not possible to embed a recursive call to {\it qtree.\/} 

For hard-to-explain reasons, trees often appear farther to the right than is
visually appealing; but not to worry, you can move them sideways by hand.
(Note the \verb|\hskip| in the next example, which moves the tree 0.5 inches
to the left).

% Here is an example
% using the {\tt enumerate} environment, and the code to generate it.

% \begin{enumerate} \item[(1)] a. \hskip-0.75in
%     \Tree[.C [.Tns [.Neg [.Ind V [.Ind Ind Agr$_1$ ]] Neg ].Neg 
%       [.Tns Agr$_2$ Tns ] ].Tns [.C C Agr$_3$ ] ]
% \end{enumerate}
% \begin{verbatim}
% \begin{enumerate} \item[(1)] a. \hskip-0.75in
%     \Tree[.C [.Tns [.Neg [.Ind V [.Ind Ind Agr_1 ]] Neg ].Neg 
%       [.Tns Agr_2 Tns ] ].Tns [.C C Agr_3 ] ]
% \end{enumerate}
% \end{verbatim}


\paragraph*{Side by side trees}
Multiple trees, or text and trees, can be arranged side by side.  This can
generally be done by just arranging commands one after another; it usually
helps to turn off tree centering.  If necessary the positioning can be
adjusted with \verb|\hskip|.
\begin{enumerate}
\qtreecenterfalse
\item[(2)] a. \hskip -0.5in\Tree [.S [.NP \'el\\he ]
        [.VP [.V hizo\\made ] [.V decir\\say ] 
             [.NP lo\\it ] [.NP a-mi\\me ]      ].VP ]
  b.  \Tree[ A [.T {B\\ \em note} cc ].T D ].S
\end{enumerate}
\label{2b}
\begin{verbatim}
\begin{enumerate}
\qtreecenterfalse
\item[(2)] a. \hskip -0.5in\Tree [.S [.NP \'el\\he ]
        [.VP [.V hizo\\made ] [.V decir\\say ] 
             [.NP lo\\it ] [.NP a-mi\\me ]      ].VP ]
  b.  \Tree[ A [.T {B\\ \em note} cc ].T D ].S
\end{enumerate}
\end{verbatim}



% \paragraph*{\it Trivia note:\/}
% If you insist on placing trees in a \verb|\parbox|, here is how to do it.
% Note the \verb|~|, called a {\it tie,\/} before \verb|\Tree| inside the
% parbox; it keeps \LaTeX\ from breaking the line and placing the tree one
% line too low.
% \begin{enumerate}
% \item \parbox[t]{2.4in}{ (a)~\Tree 
%    [.VP [.NP (mud) ] [.V\1 [.V (get) ] [.PP {(on wall)} ]]] }
% %
% \parbox[t]{2in}{ (b)~\Tree 
%  [.VP [.NP (screen) ] \qroof{V\qquad AP\\ \hfil(clear)~}.V\1 ] }
% \end{enumerate}

% \begin{verbatim}
% \begin{enumerate}
% \item \parbox[t]{2.4in}{ (a)~\Tree 
%    [.VP [.NP (mud) ] [.V\1 [.V (get) ] [.PP {(on wall)} ]]] }
% \parbox[t]{2in}{ (b)~\Tree 
%  [.VP [.NP (screen) ] \qroof{V\qquad AP\\ \hfil(clear)~}.V\1 ] }
% \end{enumerate}
% \end{verbatim}

\section{Advanced features}

\paragraph{Escaping the parser}
There is provision for sneaking directives past the front end.  If a
word begins with an exclamation mark, the next word (i.e., up to the next
space) will be passed through unchanged, except for stripping off the ``!''.
(Braces should be used to pass through larger groups).  This is mainly useful
for the manual width-adjustment directives 
\verb|\faketreewidth| and \verb|\qsetw|, described below. Note that
\verb|\qroof| should {\it not\/}  be preceded by an exclamation mark.

\paragraph{Fine tuning} The command \verb|\qsetw{<length>}| (where
\verb|<length>| might be \verb|0.5in|, \verb|36pt|, etc.) tells QobiTree to
override its default calculation of the width of the {\it just-finished\/}
node (that's the node ending just to the {\it left\/} of where the directive
was issued), and instead consider that width to be \verb|<length>|.  Similarly,
\verb|\faketreewidth{<text>}| sets the width of the last node to be equal to
the width of \verb|<text>| (which again can contain `\verb|\\|' commands etc.)
\verb|<text>| is not actually typeset but is used just to compute the fake
width of the node on the top of the stack.

% For example, the default placement rules produce the tree labeled (2b) above.
% By setting the width of the subtree headed by T to 1cm, we get the following:
% 
%  b.  \Tree[ A [.T {B\\ \em note} cc ].T !\qsetw{1cm} D ].S

For example, the default placement rules would
produce tree (a) below.  By setting the width of the subtree headed by B to
1cm, we get tree (b).

\begin{center}
\qtreecenterfalse
a. \Tree [.A [ a b c d ].B  C ]
\hfil 
b. \Tree [.A [ a b c d ].B !\qsetw{1cm} C ]
\end{center}
\begin{verbatim}
\begin{center}
\qtreecenterfalse
a. \Tree [.A [ a b c d ].B  C ]
\hfil 
b. \Tree [.A [ a b c d ].B !\qsetw{1cm} C ]
\end{center}
\end{verbatim}

When you use \verb|\qsetw| or \verb|\faketreewidth| you are on your own. They
can either shrink or enlarge the space taken by the node and may result in
trees with overlapping labels. 

\paragraph{The low-level interface}
The guts of {\it qtree\/} are the tree macros written by Jeff Siskind,
named QobiTree. Using the original interface (which is still accessible with
this package) the example tree shown on page \pageref{2b}  would be written
like this: 

\begin{center}
\leaf{A}
    \leaf{B\\ \em note} \leaf{cc} 
    \branch{2}{T}
\leaf{D}
\branch{3}{S}
\hskip 3in \qobitree
\end{center}
\vspace*{-1.5in}
\begin{verbatim}
\begin{center}
\leaf{A}
    \leaf{B\\ \em note} \leaf{cc} 
    \branch{2}{T}
\leaf{D}
\branch{3}{S}
\qobitree
\end{center}
\end{verbatim}

These macros operate like a stack machine. You push \TeX\ boxes onto the
stack of tree nodes, then you pop them off to make branching nodes which
get pushed back on the stack.


% The \emph{qtree} FAQ:
\section{How do I \ldots?}

\paragraph{Make my tree fit in the page?}  Try one or more of the following:
reduce the surrounding font size with \verb|\small| or another size command,
\emph{before} you begin the tree; reduce the amount of space between subtrees
with \verb|\qsetw| or \verb|\faketreewidth|; consider placing your tree
sideways in the page, with one of the packages that provide rotation commands.

\paragraph{Draw movement arrows from one node of a tree to another?}
Use Emma Pease's \emph{tree-dvips} package.  Despite its name it is not a very
convenient tool for creating trees, but its many line- and
arrow-drawing commands can be used to decorate trees drawn with \emph{qtree.} 


\paragraph{Use \emph{qtree} with \emph{pdflatex}?}

\emph{Pdflatex} does not support the PostScript specials generated by the
package \texttt{eepic.sty,} which \emph{qtree} loads automatically.
At present, you have the following non-ideal options:
\begin{enumerate}
\item Do not use \emph{pdflatex;}  generate a PDF file by using a
PostScript-to-PDF converter.  The disadvantage of this solution is that the 
slideshow and hypertext capabilities of PDF are not available with the
resulting files.
\item Suppress the automatic inclusion of \emph{eepic,} by using the package
option \texttt{[noeepic]}.  This unfortunately results in lower-quality
graphics, but is probably your best option if you need to use \emph{pdflatex.} 
(You will also have to do without \emph{tree-dvips} if you adopt this option).
\end{enumerate}
The ideal solution would be to develop a PDF driver for \emph{eepic,} or for
some other extention to the picture environment.  Please let me know if you
know of such a thing. 

\paragraph{Line up the text from all the leaf nodes on one horizontal line?}
As far as I can tell, \emph{qtree}'s design is incompatible with this style of
tree.  I'd love it if there was an easy way to give {\it qtree\/} this
capability (or the next one), but if there is, I haven't figured it out.

\paragraph{Draw dashed or dotted branches between certain nodes?}
Again, I can't see any way to incorporate this functionality into
\emph{qtree,} given the syntax of the front end.  You can fake it to some
extent, by creating lines that are part of a node as far as \emph{qtree} is
concerned, but which look like branch lines.


\section{Troubleshooting}
\paragraph*{Disclaimer:} This package is
distributed in the belief that it is useful in its present form.  
I welcome any comments or reports of other problems or desirable features.
But as usual, no guarantees, promises, etc.\ can be made about the present or
future state of this code.  

The following problems are not
really the fault of {\it qtree,\/} but fortunately they have easy solutions.

\paragraph{Some very short lines are not drawn}
This problem appears to be caused by the limited inventory of line slopes in
the \LaTeX\ picture environment.  For example, the tree fragment 
\verb|[.X a b ]| will produce invisible branch lines from \verb|X| to 
\verb|a| and~\verb|b|,
but the lines will reappear if the labels are made wider.  Install the picture
enhancement styles ({\tt eepic.sty\/}), and the problem will go away.

\paragraph{Qtree will not work with journal style X}

Any number of things could be going wrong, of course, but start by checking if
the journal's style redefines the {\tt tabular\/} environment.  {\it Qtree\/}
makes internal calls to {\tt tabular\/}, so this is a frequent source of
problems.  Usually the style's writer has saved the original definition of
\verb|\tabular| under a different name, so all you need to do is arrange for
the original definition to be restored for the calls to \verb|\Tree|.

There is now a hook to make this easier: If you define a command named
\verb|\qtreebugfixhook|, it will be implicitly called by \verb|\Tree|, with
local scope (so that any redefinitions it causes are automatically cancelled
at the end of the call to \verb|\Tree|).  For example, the JNLE style ({\tt
nle.sty\/}) saves the commands to begin and end a table as \verb|\oldtabular|
and \verb|\endoldtabular|, respectively, and the replacement macros result in
{\it ~r e a l l y ~ w i d e~\/} trees.  The following will restore the original
definitions for calls to \verb|\Tree| only.
\begin{verbatim}
\def\qtreebugfixhook{\let\tabular=\oldtabular 
         \let\endtabular=\endoldtabular}
\end{verbatim}
Kluwer's house style saves the original definitions as \verb|\klu@tabular| and
\verb|\klu@endtabular|, so to use {\it qtree\/} with it, do the following.
(You need the \verb|\makeatletter| call to use commands that contain an
@-sign).
\begin{verbatim}
\makeatletter
\def\qtreebugfixhook{\let\tabular=\klu@tabular 
         \let\endtabular=\klu@endtabular}
\makeatother
\end{verbatim}

\section{Inspiration}
Here is a last demo, illustrating some of the things you {\it can\/} do with
\emph{qtree.} 
(This example, and parts of the above exposition, were adapted from the
original documentation for QobiTree).

\def\CUP{{\bf cup}}
\def\Nspec{N$_{\mbox{\sc spec}}$}
\Tree [.{{\em The cup slid from John to Mary.}\\
       {GO(\CUP,  $[_{\rm Path}$ FROM({\bf John}), TO({\bf Mary})])}\\IP}
 [.\fbox{Fracture}
%
   [.{{\em The cup}\\\CUP\\NP}
     [ {{\em The}\\$\bot$\\\Nspec} {{\em cup}\\\CUP\\N} ].\fbox{Fracture}
   ]
   [.{{\em slid from John to Mary}\\
    {GO({\it x},  $[_{\rm Path}$ FROM({\bf John}), TO({\bf Mary})])}\\IP}
     [ $\vdots$ $\vdots$ !\faketreewidth{WWW} ].\fbox{Fracture}
   ]
 ].\fbox{Fracture} % repeated label
]

{\small
\begin{verbatim}
\def\CUP{{\bf cup}}
\def\Nspec{N$_{\mbox{\sc spec}}$}
\Tree [.{{\em The cup slid from John to Mary.}\\
 {GO(\CUP,  $[_{\rm Path}$ FROM({\bf John}), TO({\bf Mary})])}\\IP}
 [.\fbox{Fracture}
 %
  [.{{\em The cup}\\\CUP\\NP}
   [ {{\em The}\\$\bot$\\\Nspec} {{\em cup}\\\CUP\\N} ].\fbox{Fracture}
  ]
  [.{{\em slid from John to Mary}\\
   {GO({\it x},  $[_{\rm Path}$ FROM({\bf John}), TO({\bf Mary})])}\\IP}
   [ $\vdots$ $\vdots$ !\faketreewidth{WWW} ].\fbox{Fracture}
  ]
 ].\fbox{Fracture} % repeated label
]
\end{verbatim}} % end \small


\end{document}
