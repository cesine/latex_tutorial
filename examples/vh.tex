


\documentclass[10pt]{article}    
\usepackage{times}     
%\usepackage{doublespace}
\usepackage{vmargin}   
\usepackage{tipa}
\usepackage{covington}
\usepackage{amssymb}  
\usepackage{pifont}  
\usepackage{epsfig} 
\usepackage{epic}
\usepackage{eepic}
\newcounter{mine}
\usepackage{fullpage}
\usepackage{pstricks}
%\usepackage{ulem}  %underlining for em
\usepackage{pst-node}
%\pagestyle{empty}

%\usepackage{doublespace} %inserted for LI style purposes
%\usepackage{endnotes}   %inserted for LI style purposes

%\setstretch{1.67}      %because doublespace.sty makes too much space
%\let\footnote=\endnote

%\renewcommand{\footnote}{\endnote}  %activates endnotes.sty

\setpapersize{USletter}
\setmarginsrb{1.25in}{1.25in}{1.25in}{1.25in}{0pt}{0in}{0pt}{0in}
\begin{document}



\title{Deriving the feature-filling / 
feature-changing contrast: An application to  
 Hungarian vowel harmony\thanks{This paper has benefitted from discussion with several audiences, including the {\it Acme Balkanica Conference} at Concordia University in 2001, and  the Ohio State Linguistics Department. Especially helpful individuals have been Afton Lewis, Mark Hale, Sylvia Blaho, David Odden, Alain Th\'eriault, Peter Liem, Morris Halle, Daniela Isac and anonymous reviewers. Finally,  P\'eter Sipt\'ar deserves special thanks for vastly improving the discussion of Hungarian through careful 
reading and helpful criticism. This research was partially 
supported by the SSHRC-funded {\it Asymmetry Project} and several grants from Concordia University.  }}

\author{Charles Reiss \\ Concordia University, Montr\'{e}al}
%\date{}
\maketitle
\thispagestyle{empty}

\begin{quotation}
%This immediately raises the question of how one can judge the theoretical significance of a given fact or body of facts. The answer is that this can be done by showing that the facts in question can be accounted for as consequences of laws organized in a well articulated theory\ldots
\noindent  Unfortunately, within linguistics it has not been generally
 recognized how important such formal, theoretical work is; instead 
there is a feeling that too much concern for theoretical detail is 
a waste of time\ldots [T]he attitude that formal, theoretical work 
is bound to be both ad-hoc and sterile is, I am convinced, 
fundamentally mistaken \ldots 

\hspace{2in}Morris Halle (1975:530)
\end{quotation}





\section{Distinctness and the interpretation of structural descriptions}

 In practice, as inspection of any introductory phonology book will show, it has been implicitly assumed in generative phonology that a rule will  apply to any representation 
that contains a superset  of the information contained in the rule's structural description ($SD$). In other words, if the $SD$ of a rule $R$ subsumes a representation $Q$, then $Q$ is an input to $R$. Rules apply to natural classes of
segments, and natural classes are represented by a representation that subsumes the representation of each of its members. 

It turns out, however, that this is {\em not} the interpretive procedure 
developed in  $SPE$  (Chomsky \& Halle, 1968), the foundational work in the field:



\begin{example}\label{speip}Interpretive Procedure from $SPE$ (Chapter 8, 337)

A rule of the form $ A \rightarrow B / X {\underline{\hspace{.3in}}}Y$ applies to any string $Z = \ldots X'A'Y' \ldots$, where $X', A', Y'$ are not distinct from $X, A, Y$, respectively; and it 
converts $Z$ to  $Z' = \ldots X'B'Y' \ldots$,  where $B'$ contains all specified features of $B$ in addition to all features of $A'$ not specified in $B$.

\end{example}


\noindent Distinctness is defined as follows:\footnote{I have omitted reference to non-binary feature values.}

\begin{example}\label{dist}Distinctness in $SPE$ (336)

 `Two units U1 and U2 are distinct if and only if there is at least one feature F such that U1 is specified [$\alpha$F] and U2 is specified [$\beta$F] where $\alpha$ is plus and $\beta$ is minus\ldots'
\end{example}


\noindent The (typically implicit) appeal to subsumption 
 in general phonological practice derives from the assumption of a logical equivalence between subsumption and non-distinctness, the idea that 
if $x$ is non-distinct from $y$, then either $x$ subsumes $y$ or $y$ subsumes $x$. This equivalence does not hold, however, except under the working assumption of the $SPE$ era that representations are fully specified for all features. It {\em is} true that if either $x$ subsumes $y$ or $y$ subsumes $x$, then $x$ and $y$ are non-distinct, but a simple example can illustrate that the converse is not valid if we allow for partially specified feature matrices in lexical entries. 

%
% 1.)  I understood the above point better when we discussed it in class.  I find it confusing that you compare the SPE definition of distinctness to the implicit general phonological idea of non-distinctness.  Since SPE defines distinctness, why not state the falsely assumed definition of distinctness, instead of the falsely assumed definition of non-disticntness?  For example, ``General phonological practice assumes X is distinct from Y, iff X is not a subset of Y and Y is not a subset of X.  Clearly this is a different definition than that given in SPE.  Although, the definitions lead to the same results under the working assumption of . . .''
%

Let $x$ = [+round, -back] and let $y$ = [+round, +hi]. The representations  $x$ and  $y$ do not disagree with respect to any features, and are thus non-distinct, but one clearly does not subsume the other. And  we clearly do not expect, say, that $x$ would satisfy a  structural description specified as $y$.  

As  a further example, consider that by strict application of the $SPE$ interpretive procedure, an underspecified vowel that had only the feature [-round] would satisfy the $SD$ of a rule like (\ref{dumb}), since the representation  [-round] is not distinct from the representation [-nasal]:

%
% 2.)  It might be worth saying, ``. . . and thus are non-distinct (by SPE's definition), but . . .''
%


\begin{example}\label{dumb}
[-nasal] $\rightarrow$ [-voiced]
\end{example}


\noindent This is surely an undesirable result. 

Non-distinct representations are, in the general case, what is called `consistent' in unification-based frameworks---that is, they have no incompatible feature values. But non-distinctness, or consistency,  does not reduce to subsumption.\footnote{I have found the same point made by Bayer and Johnson (1995: Section 2) in a discussion of Lambek Categorial Grammar: `Interestingly, in cases where features are fully specified, these subsumption and consistency requirements are equivalent'. However, I do not think that the relevance of this 
observation to the application of phonological rules has been noted.}
The preceding discussion should make it clear that the interpretation of structural descriptions in generative phonology warrants reexamination.

%
% 3.) I don't know the definition of consistent.  Is it worth explaining?  In this case, I take it you mean SPE non-disticntness, and not ``mistaken'' non-distinctness, is equivalent to ``consistency''.  Also, I'm not clear on why you bring up unification-based frameworks.  It's interesting that the SPE def. of distinct is not a def. I am familiar with, after have taken Formal Methods.  I guess it must not be useful to the sort of math I studied.
%

\section{Feature counting evaluation metrics}

Perhaps all that is needed is to reject the $SPE$ interpretive procedure in favor of one appealing to subsumption, since this is what the practice has been for the last several decades.  Under this view any representation 
subsumed by
(containing a superset of the information contained in) a 
rule's Structural Description is taken to be a licit input to the rule.
This interpretive procedure has the desirable effect
 of allowing rules to apply to more than just single
 representations---they can apply to a natural class of representations 
whose description is  subsumed by  the rule's Structural Description.

This interpretive procedure entails that a rule that changed feature values, 
say from +F to -F for some feature F, would apply vacuously to representations that 
are already -F before the application of the rule.\footnote{I adopt without argument a binary-valued feature system. The paper is compatible with theories that allow various kinds of underspecification.}
For example, a straightforward statement of 
Polish or Polish  coda devoicing might be written 
as in (\ref{germ}):


\begin{example}\label{germ} $[$+cons, -son$]$ $\rightarrow$ [-voiced] in {\sc Coda} 
\end{example}

\noindent This rule  applies nonvacuously to [+voiced] inputs  that are  $[$+cons, -son$]$---in other words it makes them [-voiced]. However, according to the subsumption-based  convention of rule interpretation, the rule also applies, albeit  vacuously, to 
[-voiced] inputs  that are  $[$+cons, -son$]$.  To reiterate, both  [+voiced] and
[-voiced] can   satisfy  the SD  to be inputs to the rule. 


This interpretation of SDs is related to the {\em SPE}  feature-counting evaluation metric, the overarching goal of which is  to minimize
redundancy in the grammar, 
as seen in the Conciseness Condition formulated by Kenstowicz and Kisseberth (1979).



\begin{example}The Conciseness Condition (one component of the $SPE$ evaluation metric, from K\&K:336)

\noindent If there is more than one possible grammar that can be constructed for a given body of data, choose the grammar that is most concise in terms of the number of feature specifications.
\end{example}

\noindent With hindsight, it is now apparent that the Conciseness Condition 
is flawed by virtue of its parochialness---the model of grammar chosen by the analyst
should take into account the models necessary to generate  other languages as well as the one in question, and not just choose the most concise grammar that can generate a given corpus. Thus we can see that the Conciseness Condition as stated here is in direct conflict with the search for Universal Grammar, the grammar of S$_0$, the initial state of the language faculty:\footnote{An important question, addressed in another paper, is whether the the correct formulation of a rule is necessarily the most concise one that is consistent with the data and with the cross-linguistic (universal) demands discussed in this paper. I argue in this other work that learnability considerations provide yet another reason to favor less concise rules than we traditionally posit .}



\begin{example}\label{chom} Choosing among extensionally  equivalent grammars:\\
  ``Because evidence from Japanese can evidently bear on the  correctness of a theory of S$_0$, it can have indirect---but very powerful-- bearing on the choice of the grammar that attempts to characterize the I-language attained  by a speaker of 
English.'' [Chomsky (1986:38)]

\end{example}



 \noindent  In other words, evidence from one language should bear on the best analysis of other languages. If two  hypotheses, $\cal A$ and $\cal B$, concerning $UG$  are empirically
 adequate to provide an explanatory account  of English, but only one of the two, say $\cal A$, is  adequate  to provide an explanatory account  of Japanese, then we should 
select $\cal A$ as the best available  hypothesis for a theory of S$_0$ that can lead to acquisition of {\em  both languages} .



The   traditional  interpretation of SDs such as (\ref{germ}) is not the only 
logical possibility---it could have been argued that a rule like Polish devoicing 
should be formulated so as not to apply vacuously, as in (\ref{newgerm}):


\begin{example}\label{newgerm} $[$+cons, -son, +voiced$]$ $\rightarrow$ [-voiced] in {\sc Coda} 
\end{example}


\noindent It seems that the decision to adopt the  Conciseness Condition, and thus 
the rule format  of  (\ref{germ}), rather than (\ref{newgerm}), was motivated by the 
influence that engineering approaches to information theory had on the 
pioneers of generative phonology, an influence that has been described as leading to 
a dead end (Morris Halle, 1975:532 and p.c.). Formulation  (\ref{germ}) was seen as the more efficient, and thus better, engineering solution since it was more concise
 than  (\ref{newgerm}).\footnote{The belief that the mind organized language in a maximally efficient manner may have also motivated the Conciseness Condition. However, it could also have been argued that 
avoiding vacuous application would have constituted a more efficient solution.  
 Anderson's (1985:327) remarks on the topic are also  telling: ``Early concern for evaluation procedures \ldots turned out to be something of a dead end.'' and  ``[T]he appeal of feature counting went away \ldots not with a bang, but with a whimper.'' }


In this paper, I explore another logical possibility for the interpretation of 
SDs and show that it solves  longstanding problems in phonological theory---the 
question of 
how to allow rules to target unmarked or unspecified feature values and the intimately
related issue of the distinction between feature-filling and feature-changing rules. I then apply the solution to the generation of the alternations seen in Hungarian vowel harmony.

\section{Subsumption and  structural descriptions---a problem}
 
The SD in (\ref{germ}) subsumes various possible input representations. Crucially, all inputs
 must be specified for at least the features [+cons, -son]. For us to further understand 
the nature of  the 
set  of representations that  can serve as inputs to the rule, we need to focus on features 
that are {\em absent} from the rule's SD. The traditional understanding  of 
(\ref{germ}) depends on  two distinct interpretations of the absence of a specification:



\begin{example}\label{2interps}
\begin{itemize}
\item [a.]{Absence of a feature value  implies that the feature 
is {\em irrelevant} to the application of the rule.}
\item[b.] {Absence of a feature value  implies  that the feature 
does not need to be mentioned in the rule, because the rule neutralizes different values for the feature.}

\end{itemize}
\end{example}

\noindent The absence of reference to features for place of articulation in the input of (\ref{germ}), for example,  is interpreted as in (\ref{2interps}a) to mean that the rule applies {\em regardless} of the place of articulation of the input consonant. In other words, features such as [cor] and  [lab] do not appear in the rule because they are irrelevant to its application---Polish devoicing applies to obstruents at all places of articulation.

Assuming that  
 Polish alternating stops are underlyingly [+voiced]\footnote{I will do so without argument here. The reader will notice that if Polish alternating  stops are instead  unmarked for voicing, the problem is to voice them in the appropriate contexts without targeting the non-alternating voiceless ones.}, 
 the rule applies vacuously to underlying [-voiced] stops
 and it changes underlyingly [+voiced] stops to [-voiced]. 
Thus the rule's Structural Description (SD) contains no 
reference to [voiced] since the rule neutralizes the 
distinction between  [+voiced] and [-voiced]. This is  interpretation (\ref{2interps}b).

So,  some features are absent from the  SD because the rule does not affect 
them  or depend on them in any way (\ref{2interps}a), and others are  absent
because the rule neutralizes
 their two possible values   (\ref{2interps}b). To reiterate, any 
representation that is subsumed by the SD
  of the rule satisfies that SD.  Thus 
 (\ref{germ}) can apply to  the following inputs:

\begin{example}\label{2x2interps}
\begin{itemize}
\item [a.]{representations in which the absent features are irrelevant to rule application \hspace{.1in} {\sc and}} 
\item[b.] {representations in which the absent features are neutralized by the rule.}

\end{itemize}
\end{example}



 Consider, in contrast to the standard view of Polish-type patterns,
 a type of data  which only became 
known much later in the history of generative phonology, 
a pattern which requires rules  that fill in values on 
necessarily underspecified segments. 


In 
Turkish, for example, Inkelas (1996;  Inkelas \& Orgun 1995)  argues
 that there is necessarily a 
three-way contrast in voicing. Some stem-final stops   show a
 t/d alternation (\ref{tva}a), with [t] appearing in codas and 
[d] appearing in onsets. Inkelas  convincingly argues for an underlying segment 
 that has all the features of a coronal stop, but is unspecified for [voiced]. She  denotes this 
feature bundle as /D/. She  states that the segment is assigned the value 
[-voiced] in codas, and [+voiced] elsewhere.  
Other stem-final stops consistently surface as [t] and 
thus are posited to be /t/ underlyingly (\ref{tva}b), and others surface as [d] consistently, and are thus posited to be underlying /d/ (\ref{tva}c).  




\begin{example}\label{tva}Turkish voicing alternations\footnote{This data has been challenged in discussion on the grounds that some Turkish speakers do not pronounce the (c) forms with voiced obstruent in coda position. The irrelevance of such an objection is apparent, as long as the data represents a possible language. Since Orgun is a native speaker, I accept the data as given. The presence of inflectional morphology suggests that the forms should be  treated as  Turkish and not as French (the language they were borrowed from). I refer the reader to the cited works for further data showing that this Turkish case is not isolated---Inkelas discusses several cases with the same logical structure.  She further shows  that these data  cannot be handled by labeling certain morphemes as `exceptions'. For example, she provides examples of a single morpheme with both  an obstruent that alternates in voicing and  one that is consistently voiced (even in coda position): {\em \textipa{edZdat}} `ancestor', {\em \textipa{edZdatlar}} `ancestor-plural',   {\em \textipa{edZdadW}} `ancestor-acc.' This example shows that failure to devoice obstruents in coda position cannot be a property of individual morphemes, since the stem-medial /\textipa{dZ}/ remains voiced although it is always in coda position, whereas the stem-final /d/ alternates. Instead the two segments must be representationally distinct: the former is [+voiced] and the latter is underspecified for voicing.}
\end{example}
\begin{itemize}
\item[a.]{Alternating: [$\emptyset$voiced] (unmarked for [voiced]) /D/ \\
{\em kanat} `wing'           {\em kanatlar} `wing-plural'         {\em kanad\i m} `wing-1sg.poss'}
\item[b.]{Non-alternating voiceless: [-voiced] /t/ \\ 
{\em sanat} `art'    {\em sanatlar} `art-plural'          {\em sanat\i m} `art-1sg.poss'}
\item[c.]{Non-alternating voiced: [+voiced] /d/ \\
{\em et\"{u}d} `etude'       {\em et\"{u}dler} `etude-plural'     {\em et\"{u}d\"{u}m} `etude-1sg.poss'}
\end{itemize}



The  rule responsible for making /D/ surface as [t] in codas would be identical to (\ref{germ}), but would have to be interpreted differently, since it crucially cannot apply to underlying /d/. In other words the representation of /D/ subsumes that of /d/ (and also that of /t/), but the rule that affects /D/ does not affect /d/. It is necessary to interpret the absence of [voiced] in the SD as in (\ref{interps}c), which completes the list of interpretations of absent features under discussion: 


\begin{example}\label{interps}
\begin{itemize}
\item [a.]{Absence of a feature value  implies that the feature 
is {\em irrelevant} to the application of the rule (=\ref{2x2interps}a)}
\item[b.] {Absence of a feature value  implies  that the feature 
does not need to be mentioned in the rule, because the rule neutralizes different values for the feature (=\ref{2x2interps}b)}  
\item [c.] {Absence of a feature value implies that the feature 
{\em must} be absent from a potential input representation for the rule to apply}
\end{itemize}
\end{example}

\noindent Without an intelligent homunculus, a mental grammar 
needs a solution to the problem of correctly selecting 
the relevant  interpretation of a SD.


\section{Earlier approaches}
 One way out of this dilemma would be to allow the grammar to 
refer to [$\emptyset$voiced] as a possible specification:

\begin{example} $[$+cons, -son, $\emptyset$voiced$]$  $\rightarrow$ [-voiced] in {\sc Coda}
\end{example}

\noindent  I provide a principled argument against  allowing [$\emptyset$voiced] as a possible specification  below. Traditionally,
 this move has been avoided 
by most researchers on the intuitive grounds that it represents
 an overly powerful enrichment of the representational apparatus of phonology.

Instead, however, the notational 
 apparatus of rules has been enriched. Typically, a rule label,  {\bf Feature Filling} or, equivalently,  {\bf Structure Filling}  is  used, as in (\ref{norm}), to ensure that such a  rule is not (over-)applied
to fully specified segments and can only apply to provide feature values to underspecified segments. In the absence of such a label,  the correct interpretation is left to the intelligence of the reader.


\begin{example}\label{norm} {\bf  Feature Filling}: \hspace{.2in} $[$+cons, -son$]$ $\rightarrow$ [-voiced] in {\sc Coda}
\end{example}

\noindent  This is the solution proposed by Inkelas \& Orgun (1995:777).  I reproduce their rules exactly in (\ref{inkorg}). 



\begin{example}\label{inkorg} Feature-filling rules from Inkelas \& Orgun (1995:777)
 \begin{tabular}{l l l l }
a. & {\sc Devoicing}: & Coda plosive $\rightarrow$ [-voiced] & (structure-filling) \\
b. & {\sc Voicing}: & Onset plosive  $\rightarrow$ [+voiced] & (structure-filling) \\
\end{tabular}
\end{example}

\noindent Their rule (\ref{inkorg}a) is basically equivalent to (\ref{norm}) and they provide a second feature filling rule (\ref{inkorg}b)  to provide the alternating stops with [+voiced] in onsets.

Another example of the {\bf  Feature Filling} label can be found
 in McCarthy (1994:210), who  formulates a rule spreading [pharyngeal] from a 
consonant to a following vowel. In addition to an autosegmental representation of 
spreading, McCarthy includes the following in the rule statement: `Condition: 
Feature-filling'. He notes that the `intent of the condition restricting [the spreading]
 to feature-filling is to block the rule from lowering any vowel other than 
the featureless vowel schwa'.  Because it is featureless, consisting just of 
enough of a representation to identify it as a vowel, the
 representation of schwa
 will subsume that of every other
 vowel---less specification entails greater 
generality.








Kiparsky's (1985) discussion of coronal underspecification briefly notes
 the problem treated here, stating that it is necessary `to work out 
some way of referring to unmarked segments' when representations are not fully
 specified. However, his manner of distinguishing underspecified segments is 
not satisfactory since he introduces a new diacritic into representations just 
in places where, for example, [+coronal] (or a {\sc Coronal} class node) would be 
specified. Kiparsky proposes that (\ref{kip}a) be the representation of a coronal 
fricative such as /s/, where the x on the line to the missing node means 
`there is no specification on the tier of place features'. How do we know (or more importantly, how does the grammar `know')  that the x doesn't denote underspecification for some other feature?

\clearpage
\begin{example}\label{kip}  Representations of /s/ and /f/ (Kiparsky 1985) 


\begin{psmatrix}[rowsep=.1cm,colsep=.5cm]
a. /s/  & & &b. /f/  & \\
&    {}    && &$ \left[ \begin{array}{c}
                + \mbox{labial} \\
                \end{array}
              \right]$ \\
&    x    && &  \\
&         & &&\\
&  $ \left[ \begin{array}{c}
                + \mbox{cont} \\ 
                + \mbox{obstruent} \\
                \end{array}
              \right]$&&&$\left[ \begin{array}{c}
                + \mbox{cont} \\ 
                + \mbox{obstruent} \\
                \end{array}
              \right]$ \\
&&&& \\
& C&&&C \\
\ncline{2,2}{5,2}
%\ncline{3,2}{5,2}
\ncline{5,2}{7,2}

\ncline{2,5}{5,5}
\ncline{5,5}{7,5}
\end{psmatrix}

\end{example}


\noindent Kiparsky represents a non-coronal voiceless fricative, such as /f/ as in (\ref{kip}b) where 
the root node is associated to a labial place node. The natural class including both 
these voiceless fricatives would presumably be represented by Kiparsky as in (\ref{kip2}), 
unspecified for features on the place tier:



\begin{example}\label{kip2}  Representation of all fricatives in Kiparsky's (1985) system


\begin{psmatrix}[rowsep=.1cm,colsep=.5cm]
&&& \\
&  $ \left[ \begin{array}{c}
                + \mbox{cont} \\ 
                + \mbox{obstruent} \\
                \end{array}
              \right] $ &\\
&& \\
 & C & \\
\ncline{2,2}{4,2}



\end{psmatrix}

\end{example}


\noindent Obviously, Kiparsky's system of representing /s/ in (\ref{kip}a) is the equivalent 
of specifying [+coronal], since the representation of /s/ contains information
 not present in  (\ref{kip2}) 
and can hardly be called underspecified.\footnote{Kiparsky's suggestion leads to other problems as well---does the line 
in (\ref{kip}a) block spreading?}


Similar use of a diacritic denoting the absence of an association to a given node can be found in Archangeli (1988). Where a `melody unit or anchor' Z can be linked to a feature F {\em via} normal
 association lines between Z and F; obligatorily unlinked to F if Z is enclosed in a circle; or 
ambiguously linked or unlinked to F in the absence of an association line or circle. 


\begin{example} Linkage notation (adapted from Archangeli 1988): Z is a `melody unit or anchor
\end{example}

{\begin{psmatrix}[rowsep=.1cm,colsep=.5cm]
a.  &\pscirclebox{Z}  & &unlinked to F  \\
b. &  Z && linked to F \\
&&&\\
&F&& \\
c. &Z& &ambiguously linked or unlinked to F  \\
\ncline{2,2}{4,2} \\
\end{psmatrix}}





\noindent This notation presents a problem similar to that in Kiparsky's (1985) system, 
since the meaning of the circle around the Z in a given rule is `unlinked to the
 feature F which the rule will provide'. Thus this notation 
of `underspecification' requires reference to the very feature whose mention it is meant to avoid.




In the next section, I develop a Unified Interpretive Procedure (UIP)  for Structural Descriptions 
 that  vitiates the need for explicit {\bf  Feature Filling} or  {\bf  Feature Changing}
diacritics, as well as the need to refer to features that are absent from a representation---I do not use  [$\emptyset$F] as a possible specification. 

%In the following sections, I demonstrate the operation of this UIP by providing a new account of Hungarian 
%vowel harmony, which, I will argue is superior to others in several respects.



\section{The Unified Interpretive Procedure}

For a given rule ${\cal R}_a$, we can refer to its Structural Description as $SD_a$, its Structural Change as $SC_a$, and its Environment as $Env_a$, giving us the 
 simple rule schema in (\ref{schema}):



\begin{example}\label{schema} Rule Schema\footnote{More discussion is required for deletion, insertion and metathesis rules. These problems are addressed in work in progress. Also, the environement $Env$ is, strictly speaking, part of the $SD$, but I will treat them separately for the sake of clarity.}


{${\cal R}_a$: $SD_a$ $\rightarrow$ $SC_a$ in   $Env_a$}
\end{example}


\noindent We will employ  Greek letter variables in the usual way: $\alpha \in \{+,-\}$. 


I also assume that something like 
Principle 6 of Chomsky (1967) is valid:
``Two successive lines of a derivation can differ by at most one feature specification''.\footnote{This 
notion can be adapted for more recent theories of representation. I do not make use of, for example, feature geometry, in this paper, and I have argued elsewhere  that feature geometry
is not a necessary or desirable part of phonological theory.} Chomsky's Principle 6 helps restrict the notion of `possible phonological rule', and thus  it is desirable to conform to it. For our purposes, this principle means that  $SC_a$ will always contain a single feature specification, +F, -F or $\alpha$F.
We can now formulate the interpretive procedure for structural descriptions, replacing $\alpha$F for  $SC_a$ .
  
\begin{example}\label{uip} Unified  Interpretive Procedure for Structural Descriptions
  
A representation $Q$ is an input to  a  rule  ${\cal R}_a$: \\ \hspace{.4in}  $SD_a$ $\rightarrow$ $\alpha$F  in   $Env_a$  \\  if and only if $SD_a$ subsumes $Q$ and  one of the following holds:
\begin{itemize}
\item[a.]{ {$-\alpha$F} $\in$  $SD_a$   ($SD_a$ and thus each $Q$ that satisfies 
$SD_a$  is  specified  {$-\alpha$F})  {\sc or}}
\item[b.]{   
 $-\alpha$F $\notin$ $Q$  (no  $Q$ that satisfies 
$SD_a$  is   specified  {$-\alpha$F}, and thus neither is  $SD_a$ specified $-\alpha$F ) }  
\end{itemize}

\end{example}
%$-\alpha$F $\notin$  $SD_a$ {\sc and}


First consider (\ref{uip}a): since  (\ref{uip}a) requires that $SD_a$ be specified {$-\alpha$F}, it follows that every representation $Q$ subsumed by $SD_a$ be thus specified.   Since $SD_a$ must subsume every input to the rule, each input must also be specified {$-\alpha$F}. A representation $Q$ that satisfies this condition will undergo feature changing to {$\alpha$F}.


Now consider (\ref{uip}b): the requirement of (\ref{uip}b) is that $Q$ not be specified {$-\alpha$F}, so it can be either specified {$\alpha$F} or {\em not specified at all}  for feature F. Since $Q$ is required not to be specified $-\alpha$F, any $Q$ that satisfies this condition will not be subsumed by a  $SD_a$ which is $-\alpha$F. In other words, if $Q$ is not $-\alpha$F, and $Q$ is an input to ${\cal R}_a$, then  $SD_a$ is also not $-\alpha$F. (We thus see that (\ref{uip}a) and (\ref{uip}b) are mutually exclusive---they cannot be satisfied simultaneously.) If condition (\ref{uip}b) is fulfilled, the rule will either fill-in the  value $\alpha$F or vacuously `change' $\alpha$F to $\alpha$F.


  The two conditions thus require either (a) $Q$ is {$-\alpha$F} or (b) $Q$ is not  {$-\alpha$F}. (Further conditions are imposed by  $Env_a$, of course.)  The existence  of underspecification means that `not  {$-\alpha$F}' does not mean  `{$\alpha$F}', but rather `either {$\alpha$F} or unspecified for F'.



Thus, if $SC_a$ is  -F, and   $SD_a$  
is  specified  +F, then an input 
to ${\cal R}_a$  must contain +F in order to satisfy  $SD_a$ by condition a. However,  if $SC_a$ is  -F but  $SD_a$  
is  not specified +F, then an input 
to ${\cal R}_a$  may  not contain +F. It may be specified -F (${\cal R}_a$ will apply vacuously in this case) or it may be unspecified for F (${\cal R}_a$ will fill-in -F  in this case) and thus satisfy  $SD_a$ by condition b. 


Similarly, we can switch all the signs.  If $SC_a$ is  +F, and   $SD_a$  
is  specified  -F, then an input 
to ${\cal R}_a$  must contain -F   in order to satisfy  $SD_a$ by condition a.  However,  if $SC_a$ is   +F, but  $SD_a$  
is  not specified -F  then an input to ${\cal R}_a$  may  not contain -F, but it may be specified +F (${\cal R}_a$ will apply vacuously in this case) or it may be unspecified for F (${\cal R}_a$ will fill-in +F  in this case)  and thus satisfy  $SD_a$ by condition b.


So, (\ref{uip}a) corresponds to traditional  {\bf  Feature Changing} rules 
and  (\ref{uip}b) to traditional  {\bf  Feature Filling} rules. However, the
 UIP precludes the necessity   of referring to unmarked values such as [$\emptyset$voiced].  The crucial advance we have made is this: instead of having  
`to work out 
some way of referring to unmarked segments' we have a way
to ensure that they are treated as a class with 
representations that are vacuously affected by rules.



In a language like Polish, where we have  a two-way voiced/voiceless contrast
in obstruents we might be tempted to retain the traditional formulation of the devoicing 
rule and the traditional interpretation of structural descriptions. We would still
generate Polish-type output.  
However, a truly explanatory approach to phonology   allows us to see that Turkish can tell us something about Polish---the correct formulation of the rule must be something closer to (\ref{newgerm}) than to (\ref{germ}).  Since we are 
interested in Universal Grammar ($UG$), we are interested in a single interpretive procedure for all grammars. This was the point of the quotation from Chomsky in (\ref{chom}).

 In this particular case of the
 representation of Polish devoicing,  $UG$ should  be assumed to use the 
same interpretive procedure as is used in Turkish. More concise rules can be written   for just the Polish data, but they would not be rules of human phonology, if the UIP is correct. Since the interpretive procedure for all languages  is assumed to be identical, but the 
patterns to be accounted for are  different, the rules themselves must differ as well. Polish uses  (\ref{newgerm}), whereas Turkish uses  (\ref{germ}).


In other words, the traditional account  of Polish devoicing using the subsumption-based interpretive procedure  would be 
extensionally equivalent to the account  proposed here (using  (\ref{newgerm}) and the UIP), but we now can
choose between them in a principled fashion. Again, this  is the type of  argumentation suggested by (\ref{chom}).




A further implication of the UIP is that we now derive the intuitively valid 
 result that rules do not 
treat representations that are +F and representations that 
are -F as a natural class to the exclusion of 
representations that are unmarked for F.
With respect to the Turkish data discussed 
above this means that, for example, /t/ and /D/ constitute 
a natural class (they are {\em not} [+voiced]), and /d/ and /D/ do so as 
well (they are {\em not} [-voiced]), but that /t/ and /d/ do not, to the 
exclusion of /D/.
%\footnote{This provides an interesting  contrast to theories such as HPSG
%that recognize a value {\sc Boolean} which in combination with an attribute
  % F means, in essence, `specified as
% either +F or -F, and not unspecified for F'. It is
% my understanding that in such a 
% model, the specified segments constitute a natural class 
% to the exclusion of the unspecified ones.}





For the sake of explicitness, let us reiterate the difference between the simple subsumption-based interpretation of $SDs$    and that given by the UIP. A rule like (\ref{germ}) should apply to [+voiced] stops according the traditional interpretive procedure. 
 The $SD$  of  (\ref{germ}) is given as [+cons, -son], and since the representation of, say,  /t/, /D/ and  /d/ are subsumed by   [+cons, -son], the rule should apply to all three. But his would not let us distinguish /d/ from /D/, which we need to do for Turkish.


However, using the UIP, a rule like (\ref{germ}), where the $SC$ is [-voice],  cannot apply to 
[+voiced] representations like /d/:

\begin{itemize}
\item{ (\ref{uip}a) is not satisfied since [+voice] is not in the $SD$ of the rule as stated} 
\item{by  (\ref{uip}b) any $Q$ which is an input to the rule   cannot be specified  as  [+voice] }
  \end{itemize}


\noindent Since neither condition is satisfied, the rule cannot apply to [+voiced] representations.  However, both /t/ and /D/ satisfy  condition (\ref{uip}a), since /t/ is  [-voice] and the rule applies vacuously; and /D/  satisfies condition (\ref{uip}b), since /D/ is {\it not}  [-voice] and thus the rule fills in this value.  This would work perfectly for our model of Turkish. 




Consider now the rule in  (\ref{newgerm}), repeated here: 



\begin{example} $[$+cons, -son, +voiced$]$ $\rightarrow$ [-voiced] in {\sc Coda} 
\end{example}

\noindent Since the $SD$ of the rule contains [+voiced], the 
rule can obviously apply to [+voiced] representations. It cannot apply to representations that are
either specified [-voice] or are unspecified for [voice]---the $SD$ does 
not subsume such representations. This would work perfectly for our model of Polish. 


%\clearpage
\section{Discussion}
A reviewer of an abstract of this article  objected to the UIP on the grounds that it is an extremely powerful device, since it  is meant to 
be relevant to all rules in all languages. This objection reflects a 
misunderstanding of the notion of power in theory construction. In fact, a single interpretive procedure that holds  for all rules in all languages provides 
a {\em less} powerful (and thus better) model than one that allows various devices on an {\em ad hoc} rule-by-rule and language-by-language basis. The objection can thus be dismissed.

Note  that the result of the UIP in (\ref{uip}) can also be derived by 
allowing the use of logical negation in phonological representations. 
For example, if Turkish /t/ and /D/ were both specified 
[{\sc not}+voiced], then the rule that  fills in 
[-voiced] on /D/ in codas could refer to this specification:

\begin{example}\label{not} $[$+cons, -son, {\sc not}+voiced$]$ $\rightarrow$ [-voiced] in {\sc Coda} 
\end{example}

\noindent Allowing negation in representations to have scope over a single 
valued feature such as  $[$+voiced$]$ will not obviously create problems. However, allowing negation to have scope over sets of valued features would  wreak havoc with the notion of natural class. It would allow us to treat the complement set of each natural class as a natural class. For example, the segments described by the set {\sc not}$[$+voiced, +labial$]$ would include both [d] and [p], but not [b]. 

What the UIP does is introduce logical negation into the interpretation of rules without enriching the set of primitives that can appear in lexical representations. The UIP does redefine the notion of natural class, in fact, but in a very restricted fashion: ``[$\alpha$F]'' and ``unspecified for F'' constitute a natural class to the exclusion of ``[-$\alpha$F]'' in the sense that the difference between otherwise identical  members of a natural class ``{\sc not}[-$\alpha$F]'' is neutralized by the rule.  Under this view natural classes are not defined by a feature matrix that subsumes a set of phonological representations, but instead by a set of phonological representations that are accepted as inputs to a rule, given the UIP. In other words, natural classes are derived from the nature of rule application, rather than constituting a primitive notion  of phonological theory. I think this is a desirable result.

We can now return to the rejection of the use of 
$\emptyset$ as a coefficient value for features. Obviously, we could introduce this value, allowing the set of values to range over $\{+, -, \emptyset\}$. However, this move would have implications for the behavior of natural classes that appear to be undesirable. If phonological rules could refer to [$\emptyset$F] in structural descriptions, then it would be possible to apply rules to segments so specified without affecting other segments. For example, it should be possible to affect Turkish /D/ to the exclusion of both /d/ and /t/, say by rounding  it before round vowels:
 
\begin{example}A hypothetical rule

[$\emptyset$voiced]  $\rightarrow$ [+labial] before [+labial]
\end{example}

\noindent My intuition is that we won't find such processes, but introducing $\emptyset$ as a feature value allows such possibilities since [$\emptyset$voiced] describes a natural class to the exclusion of /d/ and /t/. 
In contrast, such a process cannot be modeled using the UIP approach. Underspecified segments can never be referred to without referring to the segments with which they neutralize on the surface. 

 Either  the introduction of negation into the set of lexical representational
primitives ($[${\sc not}+voiced$]$)  or   the use of $\emptyset$ as a possible feature coefficient ($[\emptyset$voiced$]$)  can be used to correctly model data with the logical structure of the Turkish stop alternations. However,  I  have provided  arguments that introducing logical negation into the {\em interpretive procedure for rules} is  empirically preferable to both of these alternatives. We turn now to discussion of a more complex rule type.


\section{Rules containing variables}

In the previous discussion $\alpha$ was used as a metalanguage variable to refer to either `+' or `-'
within rules. In this section, I show that UIP  applies without modification also in cases where
 $\alpha$ denotes
a variable within the rules themselves. 
 

If $SC_a$ contains the variable $\alpha$F, then 
inputs to ${\cal R}_a$  must either be specified for $-\alpha$F or else representations specified  $-\alpha$F will not satisfy the SD (following UIP).
Since $\alpha$ ranges over $+$ and $-$, $-\alpha$ also ranges over 
$+$ and $-$. This means that if the  $SC_a$  is specified for $\alpha$F,  and $SD_a$ is {\em not} specified  $-\alpha$F, then    inputs to 
  ${\cal R}_a$ cannot be specified for either +F or -F. Thus  
${\cal R}_a$ must be feature-filling---its inputs must be unspecified for F.



Immediately we see that this forces us  to revise 
 a type of rule that is commonly invoked---the type that 
involves feature changing assimilation to either -F or +F. For example, the voicing assimilation
 seen in Russian prepositions in (\ref{russ}a) is  represented in standard generative phonology as
 in (\ref{russ}b) (Halle \& Clements 1983). We see that /t/ voices to [d] and that /z/ devoices to
[s], when preceding a voiced or voiceless obstruent, respectively.


\clearpage 
\begin{example}\label{russ} Traditional statement of Russian voicing assimilation (to be 
rejected
in accordance with (\ref{uip}))

\begin{tabular}[l] {l l l l l}
 a.&   &`from'&`without'&`next to' \\ 
&`rose'&at r\'oz{\textbari} &b$^y$iz  r\'oz{\textbari}  &u  r\'oz{\textbari}  \\
&`Ala' (name) &at \'{a}l\textbari &b$^y$iz \'{a}l\textbari  &u \'{a}l\textbari \\
&`cow'&at kar\'{o}v\textbari  &b$^y$is  kar\'{o}v\textbari  &u  kar\'{o}v\textbari \\
&`beard'&ad barad\'{\textbari}  &b$^y$iz  barad\'{\textbari} &u barad\'{\textbari} \\
&`sister'&at s$^y$istr\'{\textbari}  &b$^y$is   s$^y$istr\'{\textbari} &u  
s$^y$istr\'{\textbari} \\
 &&&& \\
b.& \multicolumn{4}{l}{{$[$+cons, -son$]$ $\rightarrow$ [$\alpha$voiced] / 
{\underline{\hspace{.1in}}}  [-son, $\alpha$voiced]}}
\end{tabular}

\end{example}

\noindent According to UIP (\ref{uip}) the presence of [$\alpha$voiced] in the $SC$ and the 
lack of reference to [voiced] in the $SD$, forces the interpretation that inputs lack 
[$-\alpha$voiced]. Since $\alpha$ $\in \{+, -\}$ this rules out any value for [voiced] in the input.
Thus, in accordance with UIP,  this rule would only be licit as a feature-filling rule, applying to consonants that lacked a value for [voiced].


Assuming, as is traditional, that  Russian obstruents 
 are all specified as either [+voiced] or
 [-voiced] underlyingly, the analysis  could,
 in accordance with UIP,  be split into two rules:


\begin{example}\label{russ2} Two valued assimilation

  \begin{itemize}
  \item[a.] {$[$+voiced, +cons, -son$]$ $\rightarrow$ [-voiced] / 
{\underline{\hspace{.1in}}}  [-voiced]}
  \item[b.] {$[$-voiced, +cons, -son$]$ $\rightarrow$ [+voiced] / 
{\underline{\hspace{.1in}}}  [+voiced]}
  \end{itemize}
\end{example}



\noindent The first rule (\ref{russ2}a) would apply only to [+voiced] stops. The second  rule (\ref{russ2}b) would apply only to [-voiced] stops.

However, it  now becomes apparent that we can collapse the 
two processes and still honor the UIP.  


\begin{example}\label{russ3}
{{$[$+cons, -son, $-\alpha$voiced$]$ $\rightarrow$ [$\alpha$voiced] / 
{\underline{\hspace{.1in}}}  [-son, $\alpha$voiced]}}
\end{example}

\noindent The rule, like Polish (and Russian) coda  devoicing, is thus purely feature-changing: whenever there is feature changing, there is no vacuous application. 
If we compare  (\ref{russ3}) to (\ref{russ}b) we see that once again, 
our UIP drives us to accept a less concise representation of the 
rule. The new formulation contains three mentions of $\alpha$, as
 compared with two in the original.\footnote{Of course, it is 
common to represent such a process as a rule autosegmentally 
spreading the  [voice] 
specification  from an  obstruent to the one preceding it.  In recent work 
(Reiss, 2003ab) I argue that much of autosegmental phonology has  to be replaced by an algebraic notation 
using variables and indices. The more powerful algebraic notation is necessary,
 since phonological computation appears to require the use of quantifiers in order to express non-identity conditions. In other words, some version of 
the old-fashioned 
alpha notation may be the best way to express such processes. }


I will now apply UIP to the generation of a more complex body of data, the trigger-target relations found in Hungarian vowel harmony.



\section{Regular Patterns in  Hungarian Vowel Harmony}\label{data}
In the following discussion, I treat only the fully productive patterns of Hungarian vowel 
harmony in regular stems. For example, I leave aside the representational issues associated
 with front vowel stems that take 
back vowel suffixes, rule iterativity, 
 and also the treatment of transparent vowels. There are four  main patterns of harmonic alternations, which I first present 
with the orthographic vowels, since the orthography has typically
guided  (excessively, I believe) other treatments of this data. 
 I then turn to the phonetic values of the vowels in section \ref{vowphon}, which shows some of the deficiencies of relying too heavily on the orthography,  and then to the 
 lexical representation of alternating and non-alternating vowels in section \ref{vur}. A set of rules
for generating the surface foms from the proposed lexical representations is developed in section 
\ref{rul} and schematic derivations are given in \ref{deriv}. 

Nothing crucial relies on deriving all of these alternations from rules that correspond to `structure-filling' rules, however, the complex patterns and the relative familiarity of the data make Hungarian vowel harmony an attractive testing ground for a new proposal.

\subsection{Harmony involving {\em a/e}}

Suffixes such as inessive {\em -ban/-ben} surface as {\em -ban} 
after back vowels and {\em -ben} after front vowels: {\em dobban} `in a drum', {\em szemben} `in an eye'. These suffixes are never rounded and thus surface with the unrounded front variant also after a front round vowel: {\em t\"{o}kben} `in a pumpkin'.


\subsection{Harmony involving {\em \'{a}/\'{e}}}


 There are also alternations of the so-called `long' versions of these vowels,
 as in the translative suffix {\em -v\'{a}/-v\'{e}} (the {\em v} assimilates to a preceding consonant): : {\em dobb\'{a}} `(turn) into  a drum', {\em szemm\'{e}} `(turn) into an eye',  {\em t\"{o}kk\'{e}} `(turn) into a pumpkin'.

 
\subsection{Harmony involving {\em u/\"{u}},  {\em \'u/\H u} and {\em \'{o}/}{\em \H o} }

The other alternations with two forms always have round vowels. They agree with the backness 
of the  preceding vowel. Some consistently have a short, high rounded vowel: {\em angolul} `in English', {\em t\"{o}r\"{o}k\"{u}l} `in Turkish', {\em lengyel\"{u}l} `in Polish'. Others have the long counterparts of these high vowels: {\em l\'ab\'u} `-legged', {\em fej\H u} `-headed'.  Finally, one set consistently
 has a 
long, mid rounded vowel: {\em dobt\'{o}l} `from a drum', 
{\em szemt\H ol} `from an eye', 
 {\em t\" okt\H ol} `from a pumpkin'. Thus, the height of the preceding vowel never affects the harmonic vowel:
 {\em t\H uzt\H ol} `from a fire'.


\subsection{Harmony involving {\em e/\"{o}/o}}


Finally we must consider harmony patterns with three alternants. These are restricted to
 a single pattern involving 
mid vowels: {\em e/\"{o}/o}.  This pattern is seen in the superessive suffix   {\em -en/-\"{o}n/-on}:  {\em szemen} `on an eye', {\em t\"{o}k\"{o}n} `on a pumpkin', {\em dobon} `on a drum'.\footnote{The superessive does also surface as just {\it -n} when the stem ends in a vowel, as in {\it kapu-n} `on a gate'. I assume this results from a straightforward vowel deletion rule.} 
The vowel of this suffix  is present underlyingly, and it can be accounted for straightforwardly. 


However, there are other vowels that participate in a vowel-zero alternation and may be analyzed 
as epenthetic. These vowels generally show the same {\em e/\"{o}/o} three-way contrast. 
 Such is the case of the accusative suffix: {\em szemet} `eye-{\sc Acc}',
 {\em t\"{o}k\"{ot} `pumpkin-{\sc Acc}', {\em dobot} `drum-{\sc Acc}'}.
These `unstable' vowels 
 also surface as {\em a/e} after so-called `lowering stems'. 
I leave a discussion of such stems aside for the purposes of this paper.




\section{Vowel Phonetics}\label{vowphon}

In the preceding section I  referred to the vowels using standard orthography and 
occasional reference to their phonetic features. In this section I provide the IPA 
symbols that represent the vowels in standard pronunciation, along with a featural 
representation. The umlaut marks front rounded vowels. 
The lengthened umlaut marks long front rounded vowels, and an  acute accent
marks all the other  long vowels. Thus, one of the two orthographic marks of 
length is always present on a phonetically long vowel, and never present 
(aside from a few frozen spellings) on short vowels. 
However, these vowels do not always differ from their `short' counterparts 
in length alone---they may also differ featurally, with respect to height, rounding or tenseness.

The IPA symbols and featural descriptions I have assigned are based on 
examination of a variety of sources, most importantly consultation with native speaker linguists P\'eter Sipt\'ar and  Sylvia Blaho. 

Many discussions 
of Hungarian vowels abstract away from predictable and 
non-distinctive features. For example, 
Sipt\'ar (1994) says   of the {\it \'a} vowel that the height (extra low, in 
comparison with  {\it a/e}, 
which are often treated as low) and backness (fairly central) is ``a
matter of phonetic implementation since in the (morpho)phonological pattern of Hungarian [this vowel] behaves as a low back vowel ({\it e.g.} with respect to vowel harmony, long/short alternations, {\it etc.})'' (175).  
I will assume a minimal amount of underspecification (following Inkelas 1996), 
and thus include such features. 
Assuming an  inventory of binary vowel features that consists of \{hi, lo, bk, rd, ATR\} I prefer to remain faithful  to the phonetic facts wherever possible  and to show how 
doing so helps us to account for  a complex set of interactions. 

%\clearpage

\begin{example}\label{surf}Surface vowels of Hungarian

\begin{tabular}[c]{c l l l}
orthography & IPA & features & length  \\ \hline
{\em i} & [\textipa{i}]& [+hi, -lo, -bk, -rd, +ATR]  & {\sc short}  \\
{\em \'\i} & [\textipa{i:}]& [+hi, -lo, -bk, -rd, +ATR] & {\sc long} \\ 
{\em \"{u}} & [\textipa{y}]& [+hi, -lo, -bk, +rd, +ATR] & {\sc short}  \\
{\em \H u} & [\textipa{y:}]& [+hi, -lo, -bk, +rd, +ATR]  & {\sc long}\\
{\em e} & [\textipa{E}]& [-hi, -lo, -bk, -rd, -ATR] & {\sc short}\\
{\em \'{e}} & [\textipa{e:}]& [-hi, -lo, -bk, -rd, +ATR]  & {\sc long}\\
{\em \"{o}} & [\textipa{\o}]& [-hi, -lo, -bk, +rd, +ATR] & {\sc short}\\
{\em \H o} & [\textipa{\o:}]& [-hi, -lo, -bk, +rd, +ATR]  & {\sc long}\\
{\em u} & [\textipa{u}]& [+hi, -lo, +bk, +rd, +ATR ] & {\sc short}\\
{\em \'u} & [\textipa{u:}]& [+hi, -lo, +bk, +rd, +ATR]  & {\sc long}\\
{\em o} & [\textipa{o}]& [-hi, -lo, +bk, +rd, +ATR] & {\sc short}\\
{\em \'o{}} & [\textipa{o:}]& [-hi, -lo, +bk, +rd, +ATR]  & {\sc long}\\
{\em a} & [\textipa{O}]& [-hi, -lo, +bk, +rd, -ATR] & {\sc short}\\
{\em \'{a}} & [\textipa{a:}]& [-hi, +lo, +bk, -rd, +ATR]  & {\sc long}\\

\end{tabular}
\end{example}

\noindent  Note that all the high vowels are [+ATR]. For   the mid, front, unrounded vowels the short one is lax ([-ATR]),
 the long one tense  ([+ATR]). However, the other four mid vowels, all 
the round ones,  are all tense. Finally, note that 
orthographic {\em a} represents [\textipa{O}], a mid, back, round, lax vowel, whereas {\em \'{a}} 
is low and unrounded and tense.
 


Most analyses of Hungarian abstract away from all of these surface distinctions, 
for example, treating {\em a} as an unrounded vowel for the purposes of harmony. In the following,
 I follow a different course.




\section{Vowels in Underlying Representation}\label{vur}


Following recent work such as Inkelas (1996) and others, I assume 
that underspecification is only posited by a learner, when it is forced to do so by 
alternations in the data. In other words, it is logically possible that Polish has /t/ {\it vs.\ } /D/, rather than /t/ {\it vs.\ } /d/, but we assume that 
/D/ is only posited if /t/ and /d/ are `already assigned'. The child's default parse is that underlying segments are identical to surface segments, and this default is rejected only if alternations force a new analysis\footnote{See Hale \& Reiss (1998) for arguments concerning the transparency of the child's initial parse.}, as in Turkish. Polish does not present the child with data that will force such a change. Since the transparent parse works, the learner sticks with it and does not consider other extensionally equivalent grammars.\footnote{An additional source of underspecification may arise in order to 
capture distributional patterns, but the point here is that there 
is no obvious need to minimize the amount of information stored in lexical representations, since this represents a longer learning path---see Hale \& Reiss (2000) for discussion.}



\begin{example}\label{iou}Inkelas on underspecification
\end{example}

\begin{quotation}
\noindent{underlying representation is determined solely by optimization with respect to the 
grammar, not by imposing any type of constraints directly on underlying representation\ldots\ [this] 
results in the use of underspecification only when there are alternant surface forms\ldots\ 
(Inkelas 1996:1).}
\end{quotation}

\noindent Inkelas specifically rejects philosophical arguments for underspecification:

\begin{example}\label{gb}`Grammar-blind' approaches to underspecification rejected 
by Inkelas 1996, q.v.\ for 
references
\end{example}


\begin{itemize}
\item{{\bfseries Markedness} (universal, language-specific, or contextual); unmarked material is 
underspecified}
\item{{\bfseries Redundancy}; redundant feature values (determined on the basis of the segment 
inventory) are underspecified} 
\item{{\bfseries Predictability}: predictable material is underspecified}
\end{itemize} 

\noindent According to Inkelas, `[t]he only motivation for underspecification is to capture alternations in 
the optimal way' (1996: 2). The underspecification posited for the harmonic vowels of Hungarian will be guided by this principle.





\subsection{Nonalternating vowels}

In the case of non-alternating vowels, I assume that they enter and exit the phonology with the 
same featural representations. There are no redundancy rules, for example. Thus, 
 both nonalternating and alternating high  vowels are [+ATR] in the lexicon, and they surface as
 [+ATR]. The fact that [+ATR] is predictable from [+hi] is not encoded in the grammar.





\subsubsection{Stem vowels}

Aside from a few length alternations and vowel-zero alternations that will not concern us, all 
stem (and prefix) vowels in Hungarian have a constant value.\footnote{Of course, a full 
evaluation of my proposal will ultimately require accounting for the length alternations 
as well. I leave  this for future research.} All of the vowels in (\ref{surf}) 
appear in stems, and many in prefixes. I assume that they all have the underlying and surface 
featural representations given above. 


\subsubsection{Suffix  vowels}

Vowel harmony in Hungarian only affects suffixes. However, not all 
suffixes are subject to harmony. The following suffixes are among those that 
do not alternate: {\em -ig} `until', {\em -k\'{e}nt} `as', {\em -kor} `at a time': {\em hatkor} `at six o'clock', {\em h\'{e}tkor} `at seven o'clock'.


\subsection{Alternating vowels}

I now propose underlying representations for the alternating harmonic vowels. I will assume that 
each of the alternations mentioned in section \ref{data} can be captured {\em via} a
partially  underspecified representation and a set of phonological rules that {{\bf fill in} some feature values---no feature changing rules are needed. Length is mentioned in the prose, but not represented featurally, since I assume it is represented through multiple linking to a timing tier.\footnote{The 
careful reader will have noted that I earlier 
(fn.11) rejected some uses of autosegmental phonology, but am here appealing to an  autosegmental representation of length. This will have to be dealt with.}
.}
 

\subsubsection{The lexical form of the  {\em a/e} vowel}

Suffixes such as the inessive  {\em -ban/-ben}, illative {\em -ba/-be} and sublative 
 {\em -ra/re} are all stored with the following feature values for their 
alternating vowel: [-hi, -lo, -ATR]. This should not be controversial, 
since such suffixes do in fact alternate with respect to [bk] and [rd] ({\em a} is [+round]).
The specification [-lo] is maintained, since both realizations of the vowel, [\textipa{O, E}], 
are mid vowels.

\subsubsection{The lexical form of the  {\em \'{a}/\'{e}} vowel}

These vowels are never high or round, but they alternate between front mid and low back. I
assume that both forms are [+ATR], although this may be debatable. Accordingly, the underlying form is [-hi, -rd, +ATR].


\subsubsection{The lexical form of the consistently rounded vowels}
These  alternations can all be generated  by the same rule---the only alternating feature is [bk].

\begin{itemize}
\item{ {\em u/\"{u}} \\
This alternation is between a front and back lax high round vowel. It is stored as 
[+hi, -lo, +rd, +ATR], with no specification for [bk]. This feature is filled in by context.}


\item{ {\em \'u/\H u}\\
This alternation is between long vowels that are consistently high, round and tense. These vowels are represented as [+hi, -lo, +rd, +ATR].}

\item {{\em \'{o}/\H o}\\
This alternation is between long vowels that are consistently mid, round and tense. These vowels are represented as [-hi, -lo, +rd, +ATR].}


\end{itemize}


\subsubsection{The lexical form of the  {\em e/\"{o}/o} vowel}

The most complex alternation to deal with is that which shows three surface
 forms. As mentioned above, the presence of a  vocalic feature bundle is  
 underlying in some cases of three-way alternations and  epenthetic in others. Of course, in the latter case,
it is misleading to refer to their lexical representation. However, I will assume that both 
underlying and epenthetic three-form suffixes enter the phonology as [-hi, -lo]
 `Feature-filling' rules will provide the values for [bk] and [rd] and [ATR].


\subsection{Personal pronoun forms}

Corresponding to the case endings discussed above are a set of stems that 
inflect for person. For example, the  inessive  {\em -ban/-ben} corresponds to
{\it bennem,  benned,  benne,    benn\"unk,  bennetek,  benn\"uk} `in me, in you \ldots'. The stem appears with the front vowel [\textipa{E}] with which the person 
endings harmonize. Similarly, the delative  {\em -r\'ol/-/r\H ol} 
corresponds to 
{\it r\'olam,      r\'olad,        r\'ola, r\'olunk, r\'olatok,   r\'oluk} 
 `from off of me, from off of  you \ldots', all with back vowel suffixes.
 It is thus tempting to see these case stems as providing evidence for the
 underlying vowel quality of the case suffixes. 

However, in several categories,
 the case stem and case suffix do {\it not} match phonologically. 
 The superessive ending {\it -en, -on, -\"on} corresponds to the  case stem in {\it  rajtam,  rajtad,  rajta, rajtunk,  rajtatok, rajtuk} `on me, on you \ldots'. The elative   {\em -b\'ol/-/b\H ol}  corresponds to the  case stem in 
 {\it bel\H olem, bel\H oled, bel\H ole,  bel\H ol\"unk,     bel\H oletek, 
   bel\H ol\"uk}  `from in me, from in you \ldots'. Thus I will assume 
 that the 
case endings and the case stems are listed independently in the lexicon. 
In other words, {\it contra} Ringen \& Vago (1998), I conclude that 
there is a set of  suffixes, such as /-bVn/, with underspecified, alternating vowels, and a set of stems such as /b{\textipa E}n-/ with  fully specified,
 nonalternating vowels.  In some instances ({\it e.\ g.\ }, inessive /-bVn/ and /b{\textipa E}n-/), the forms are clearly 
 related {etymologically}, whereas in others   ({\it e.\ g.\ }, elative/-bVl/ and  /bel\H ol-/,  or  superessive /-Vn/ and /rajt-/) the 
relationship is opaque. The case stems contain fully specified vowels that 
trigger harmony in the personal suffixes; the case endings harmonize with the stems they attach to---there is no need to posit feature-changing processes
in those instances where the case stem and case suffix just happen to 
look alike.


\section{The Rules}\label{rul}



In this paper, I am concerned only with capturing the interactions between
 triggers and targets of harmony. 
Therefore, I will not address the issue of transparent vowels or of roots with surface front vowels that trigger back harmony. I will also not propose an explicit  mechanism for  harmony in terms of syllable structure or higher-level structures. 
 I will adopt a simplified
 view in which  harmony occurs between a vowel and the vowel that precedes it.
  The following
seven  rules are posited with crucial ordering noted. All seven are what would
be called feature-filling or structure-filling  rules in other accounts.\footnote{Some  also happen to be what are traditionally called fill-in rules---rules that supply default values only with reference to context within the segment itself.}
This does not need to be stated for each rule, since it follows from the UIP. 


\begin{itemize}
\item[${\cal R}_1$:] {V  $\rightarrow$ [$\alpha$bk] / 
  [$\alpha$bk] {\underline{\hspace{.1in}}} \\ This rule 
applies first. According to the UIP, it can only apply to 
vowels that have no specification for [bk]. Thus it applies 
to all of the alternating vowels and none of the non-alternating ones. }

\item[${\cal R}_2$:]   {[-lo, +bk] $\rightarrow$ [+rd] \\ 
 In accordance with the UIP, this rule  applies to  [-lo, +bk] that are {\sc not} [-rd]. The rule applies 
vacuously to already rounded vowels that are also [-lo, +bk], and in feature filling fashion to [-lo, +bk] vowels that are unspecified for [rd].   Since ${\cal R}_2$ applies 
 to vowels which became [+bk] {\em via} ${\cal R}_1$,  ${\cal R}_2$ 
must follow ${\cal R}_1$. 
For example, a vowel that enters the  phonology as [-hi, -lo] could 
 become [+bk] by ${\cal R}_1$ if it follows a back vowel,  and then [+rd] by ${\cal R}_2$. It would then receive and ATR value from ${\cal R}_6$ and at this point it would be fully 
specified for all the vowel features.


\item [${\cal R}_3$:]   {[+bk, -rd] $\rightarrow$ [+lo] \\ This rule ensures 
that {\em \'{a}} is realized as a low vowel, the only one in the language. Crucially, the 
vowel of the {\em \'{a}/\'{e}} alternation is underlyingly specified as [-rd]. The rule refers to [+bk], and since all the alternating vowels are underlyingly unspecified for [bk],
 ${\cal R}_3$  must follow ${\cal R}_1$.}


\item [${\cal R}_4$:]  {[-hi, -bk, -ATR] $\rightarrow$ [-rd] \\ This rule ensures that the 
{\em e} of the {\em e/a} alternation surfaces as unround. It cannot effect this 
feature insertion for the {\em e} of the  {\em e/\"{o}/o} alternation, since the latter is not
specified as [-ATR] until ${\cal R}_6$ applies.}



\item [${\cal R}_5$:]  {V $\rightarrow$ [$\alpha$rd] /  [$\alpha$rd] {\underline{\hspace{.1in}}} \\ By the UIP, this rule 
cannot apply to any vowel that is already specified for rounding. It turns out that it only ends up affecting  the front vowels of the 
{\em e/\"{o}/o}  alternation. All other  vowels that receive rounding by rule get [+rd] from  ${\cal R}_2$ or 
 ${\cal R}_4$. Thus, ${\cal R}_5$ follows these two. }


\item [${\cal R}_6$:]  {[-hi, -lo, $\alpha$rd]  $\rightarrow$ [$\alpha$ATR]} 
\\ This rule makes mid vowels that are unspecified for [ATR]
 receive an [ATR] value that agrees with their rounding value. The UIP
 requires that inputs to this rule not have an [ATR] value, thus there is no chance that  long {\em \'{e}} will be affected by this rule. This rule applies after ${\cal R}_2$ and also after ${\cal R}_5$, since these fill in the 
values of [rd] required for the rule to apply.


\item [${\cal R}_7$:]  {[-hi, -bk]  $\rightarrow$ [-lo] \\ This provides the [-lo] value to the front member of the {\em \'{a}/\'{e}} alternation. Since the SD refers to [bk], ${\cal R}_7$ 
must follow ${\cal R}_1$. The rule will apply vacuously to a derived [-hi, -bk, -lo] {\em e}, as well. }

\end{itemize}



\section{Derivations}\label{deriv}


The following tables provide derivations of each of the alternations. Input and output are given with orthographic vowels.  The only features of the trigger of harmony that are relevant are [bk] and [round], 
so it is not necessary to provide all trigger-target pairs. For example, the vowel {\em u} will trigger the exact same harmony patterns as {\em o}. I have provided schematic derivations for four trigger vowels  
representing  each of the four possible 
combinations of these features: [+bk, -rd], [+bk, +rd], [-bk, -rd] and [-bk, -rd]. 

In the first table (\ref{a/e}), I have annotated the cases where the rule does not affect the input, 
whereas in the subsequent tables I merely note `N.A.' for `not affected'. 
If a box is marked 
NS (no subsumption), it means that the $SD$ of
 the rule does not subsume the potential input representation.  For example, ${\cal R}_3$ does not apply to any forms in table (\ref{a/e}), since the structural description of ${\cal R}_3$ does not subsume any of the input 
representations  in this table.
Thus, such cases conform to standard practice---the rule does not apply because it does not match the $SD$. 

However, some cells are marked UIP to show that 
the current rule's $SD$ would be met under traditional assumptions, 
but it is not met under the UIP. For example,  ${\cal R}_5$ does not apply 
to any forms in (\ref{a/e}) since all the inputs are specified for [rd]
by this point in the derivation, and  ${\cal R}_5$ must be 
feature-filling by the UIP.


Finally, some cells are marked VAC, 
where the $SD$  is satisfied under the UIP, but the effect of the rule 
is not visible (it applies vacuously). This is the case for  ${\cal R}_7$ in the second and fourth data columns in (\ref{a/e}), where it applies vacuously to vowels that are already [-hi, -lo, -bk]. The only rules that affect the output are  ${\cal R}_1$ (for all forms in this table),  ${\cal R}_2$ (forms preceded by a back vowel trigger), and  ${\cal R}_4$ (forms preceded by a front vowel trigger). 



%\clearpage

\begin{example}\label{a/e}Generating the {\em a/e} alternation
  
\end{example}
\footnotesize
\noindent \begin{tabular}[l]{||c| c| c| c| c||} \hline \hline 
 & \'{a} {\underline{\hspace{.1in}}} & e {\underline{\hspace{.1in}}} & o {\underline{\hspace{.1in}}} & \"{o} {\underline{\hspace{.1in}}} \\ 
In.  &$\left[ \begin{array}{c}
                - \mbox{hi} \\ 
                + \mbox{lo} \\
                + \mbox{bk} \\
               -  \mbox{rd} \\
               +  \mbox{ATR} \\
                \end{array}
                \right]$ 
 $\left[ \begin{array}{c}
                - \mbox{hi} \\ 
                - \mbox{lo} \\
                -  \mbox{ATR} \\
                \end{array}
              \right]$  &
                $\left[ \begin{array}{c}
                - \mbox{hi} \\ 
                - \mbox{lo} \\
                - \mbox{bk} \\
               -  \mbox{rd} \\
               -  \mbox{ATR} \\
                \end{array}
                \right]$  
 $\left[ \begin{array}{c}
                - \mbox{hi} \\ 
                - \mbox{lo} \\
                -  \mbox{ATR} \\
                \end{array}
              \right]$  &
                $\left[ \begin{array}{c}
                - \mbox{hi} \\ 
                - \mbox{lo} \\
                + \mbox{bk} \\
               +  \mbox{rd} \\
               +  \mbox{ATR} \\
                \end{array}
                \right]$ 
$\left[ \begin{array}{c}
                - \mbox{hi} \\ 
                - \mbox{lo} \\
                -  \mbox{ATR} \\
                \end{array}
              \right]$ &
               $\left[ \begin{array}{c}
                - \mbox{hi} \\ 
                - \mbox{lo} \\
                - \mbox{bk} \\
               +  \mbox{rd} \\
               +  \mbox{ATR} \\
                \end{array}
                \right]$
$\left[ \begin{array}{c}
                - \mbox{hi} \\ 
                - \mbox{lo} \\
                -  \mbox{ATR} \\
                \end{array}
                \right]$   \\ \hline 
${\cal R}_1$ &--------------- $\left[ \begin{array}{c}
                - \mbox{hi} \\ 
                - \mbox{lo} \\
                + \mbox{bk} \\
                -  \mbox{ATR} \\
                \end{array}
              \right]$&--------------- $\left[ \begin{array}{c}
                - \mbox{hi} \\ 
                - \mbox{lo} \\
                - \mbox{bk} \\
                -  \mbox{ATR} \\
                \end{array}
              \right]$&--------------- $\left[ \begin{array}{c}
                - \mbox{hi} \\ 
                - \mbox{lo} \\
                 + \mbox{bk} \\
                -  \mbox{ATR} \\
                \end{array}
              \right]$& ---------------  $\left[ \begin{array}{c}
                - \mbox{hi} \\ 
                - \mbox{lo} \\
                - \mbox{bk} \\
                -  \mbox{ATR} \\
                \end{array}
              \right]$  \\ \hline 
${\cal R}_2$ & --------------- $\left[ \begin{array}{c}
                - \mbox{hi} \\ 
                - \mbox{lo} \\
                + \mbox{bk} \\
                + \mbox{rd} \\
                -  \mbox{ATR} \\
                \end{array}
              \right]$& NS & --------------- $\left[ \begin{array}{c}
                - \mbox{hi} \\ 
                - \mbox{lo} \\
                 + \mbox{bk} \\
                  + \mbox{rd} \\
                -  \mbox{ATR} \\
                \end{array}
              \right]$& NS \\ \hline 
${\cal R}_3$ &NS&NS&NS&NS \\ \hline

${\cal R}_4$ &NS&--------------- $\left[ \begin{array}{c}
                - \mbox{hi} \\ 
                - \mbox{lo} \\
                 - \mbox{bk} \\
                  - \mbox{rd} \\
                -  \mbox{ATR} \\
                \end{array}
              \right]$&NS& --------------- $\left[ \begin{array}{c}
                - \mbox{hi} \\ 
                - \mbox{lo} \\
                 - \mbox{bk} \\
                  - \mbox{rd} \\
                -  \mbox{ATR} \\
                \end{array}
              \right]$ \\ \hline 
${\cal R}_5$ &UIP&UIP&UIP&UIP \\ \hline
${\cal R}_6$ &UIP&UIP&UIP&UIP \\ \hline
${\cal R}_7$ &NS&VAC&NS&VAC \\ \hline 
Out.  &{\em \'{a}-a}&{\em e-e}&{\em o-a}&{\em \"{o}-e} \\ \hline  \hline 
\end{tabular} 


\normalsize
\vspace{.4in}

\clearpage


In (\ref{longae}) we derive the {\em \'{a}/\'{e}} alternation. We see that rules ${\cal R}_1$ and  ${\cal R}_3$ affect  the forms triggered by preceding back vowels, whereas  rules ${\cal R}_1$ and  ${\cal R}_7$ affect  the forms triggered by preceding front vowels.



\begin{example}\label{longae}Generating the {\em \'{a}/\'{e}} alternation
  
\end{example}
\footnotesize
\noindent \begin{tabular}[l]{||c| c| c| c| c||} \hline \hline 
 & \'{a} {\underline{\hspace{.1in}}} & e {\underline{\hspace{.1in}}} & o {\underline{\hspace{.1in}}} & \"{o} {\underline{\hspace{.1in}}} \\ 
In.  &$\left[ \begin{array}{c}
                - \mbox{hi} \\ 
                + \mbox{lo} \\
                + \mbox{bk} \\
               -  \mbox{rd} \\
               +  \mbox{ATR} \\
                \end{array}
                \right]$ 
 $\left[ \begin{array}{c}
                - \mbox{hi} \\ 
                - \mbox{rd} \\
                +  \mbox{ATR} \\
                \end{array}
              \right]$  &
                $\left[ \begin{array}{c}
                - \mbox{hi} \\ 
                - \mbox{lo} \\
                - \mbox{bk} \\
               -  \mbox{rd} \\
               -  \mbox{ATR} \\
                \end{array}
                \right]$  
 $\left[ \begin{array}{c}
                  - \mbox{hi} \\ 
                - \mbox{rd} \\
                +  \mbox{ATR} \\
                \end{array}
              \right]$  &
                $\left[ \begin{array}{c}
                - \mbox{hi} \\ 
                - \mbox{lo} \\
                + \mbox{bk} \\
               +  \mbox{rd} \\
               +  \mbox{ATR} \\
                \end{array}
                \right]$ 
$\left[ \begin{array}{c}
                 - \mbox{hi} \\ 
                - \mbox{rd} \\
                +  \mbox{ATR} \\
                \end{array}
              \right]$ &
               $\left[ \begin{array}{c}
                - \mbox{hi} \\ 
                - \mbox{lo} \\
                - \mbox{bk} \\
               +  \mbox{rd} \\
               +  \mbox{ATR} \\
                \end{array}
                \right]$
$\left[ \begin{array}{c}
                - \mbox{hi} \\ 
                - \mbox{rd} \\
                +  \mbox{ATR} \\
                \end{array}
                \right]$   \\ \hline 
${\cal R}_1$ &--------------- $\left[ \begin{array}{c}
                - \mbox{hi} \\ 
               - \mbox{rd} \\
                + \mbox{bk} \\
                +  \mbox{ATR} \\
                \end{array}
              \right]$&--------------- $\left[ \begin{array}{c}
                - \mbox{hi} \\ 
               - \mbox{rd} \\
                - \mbox{bk} \\
                +  \mbox{ATR} \\
                \end{array}
              \right]$&--------------- $\left[ \begin{array}{c}
                - \mbox{hi} \\ 
                 - \mbox{rd} \\
                 + \mbox{bk} \\
                +  \mbox{ATR} \\
                \end{array}
              \right]$& ---------------  $\left[ \begin{array}{c}
                - \mbox{hi} \\ 
                - \mbox{rd} \\
                - \mbox{bk} \\
                +  \mbox{ATR} \\
                \end{array}
              \right]$  \\ \hline 
${\cal R}_2$  &N.A.&N.A.&N.A.&N.A. \\ \hline
${\cal R}_3$ &--------------- $\left[ \begin{array}{c}
                - \mbox{hi} \\ 
		 + \mbox{lo} \\
               - \mbox{rd} \\
                 + \mbox{bk} \\
                +  \mbox{ATR} \\
                \end{array}
              \right]$
&N.A.
&--------------- $\left[ \begin{array}{c}
                - \mbox{hi} \\ 
                + \mbox{lo} \\
                  - \mbox{rd} \\
                 + \mbox{bk} \\
                +  \mbox{ATR} \\
                \end{array}
              \right]$
& N.A.  \\ \hline 

${\cal R}_4$  &N.A.&N.A.&N.A.&N.A. \\ \hline
%${\cal R}_passe$ &N.A.&N.A.&N.A.&N.A. \\ \hline 
${\cal R}_5$ &N.A.&N.A.&N.A.&N.A. \\ \hline
${\cal R}_6$ &N.A.&N.A.&N.A.&N.A. \\ \hline
${\cal R}_7$  &N.A.&--------------- $\left[ \begin{array}{c}
                - \mbox{hi} \\ 
                 - \mbox{lo} \\ 
               - \mbox{rd} \\
                - \mbox{bk} \\
                +  \mbox{ATR} \\
                \end{array}
              \right]$&N.A.& ---------------  $\left[ \begin{array}{c}
                - \mbox{hi} \\
                  - \mbox{lo} \\ 
                - \mbox{rd} \\
                - \mbox{bk} \\
                +  \mbox{ATR} \\
                \end{array}
              \right]$  \\ \hline 
Out.  &{\em \'{a}-\'{a}}&{\em e-\'{e}}&{\em o-\'{a}}&{\em \"{o}-\'{e}} \\ \hline  \hline 
\end{tabular} 

\normalsize




\vspace{.4in}




%\clearpage

The next two  tables (\ref{u},  \ref{longo}) show the derivation
of front/back alternations of consistently round vowels. In both  cases the only relevant rule is ${\cal R}_1$. Since the long high round vowels {\it {\H u}, \'u} behave in completely parallel fashion to the short ones {\it \"u, u}, we show only the latter.

\clearpage
\begin{example}\label{u}Generating the {\em u/\"{u}} alternation
  
\end{example}
\footnotesize
\begin{tabular}[l]{||c| c| c| c| c||} \hline \hline 
 & \'{a} {\underline{\hspace{.1in}}} & e {\underline{\hspace{.1in}}} & o {\underline{\hspace{.1in}}} & \"{o} {\underline{\hspace{.1in}}} \\ 
In.  &$\left[ \begin{array}{c}
                - \mbox{hi} \\ 
                + \mbox{lo} \\
                + \mbox{bk} \\
               -  \mbox{rd} \\
               +  \mbox{ATR} \\
                \end{array}
                \right]$ 
 $\left[ \begin{array}{c}
                + \mbox{hi} \\ 
                - \mbox{lo} \\
                + \mbox{rd} \\
                +  \mbox{ATR} \\
                \end{array}
              \right]$  &
                $\left[ \begin{array}{c}
                - \mbox{hi} \\ 
                - \mbox{lo} \\
                - \mbox{bk} \\
               -  \mbox{rd} \\
               +  \mbox{ATR} \\
                \end{array}
                \right]$  
 $\left[ \begin{array}{c}
                   + \mbox{hi} \\ 
                - \mbox{lo} \\
                + \mbox{rd} \\
                +  \mbox{ATR} \\
                \end{array}
              \right]$  &
                $\left[ \begin{array}{c}
                - \mbox{hi} \\ 
                - \mbox{lo} \\
                + \mbox{bk} \\
               +  \mbox{rd} \\
               +  \mbox{ATR} \\
                \end{array}
                \right]$ 
$\left[ \begin{array}{c}
                 + \mbox{hi} \\ 
                - \mbox{lo} \\
                + \mbox{rd} \\
                -  \mbox{ATR} \\
                \end{array}
              \right]$ &
               $\left[ \begin{array}{c}
                - \mbox{hi} \\ 
                - \mbox{lo} \\
                - \mbox{bk} \\
               +  \mbox{rd} \\
               +  \mbox{ATR} \\
                \end{array}
                \right]$
$\left[ \begin{array}{c}
                + \mbox{hi} \\ 
                - \mbox{lo} \\
                + \mbox{rd} \\
                +  \mbox{ATR} \\
                \end{array}
                \right]$   \\ \hline 
${\cal R}_1$ &--------------- $\left[ \begin{array}{c}
                + \mbox{hi} \\ 
                - \mbox{lo} \\
                + \mbox{bk} \\
                 + \mbox{rd} \\
                +  \mbox{ATR} \\
                \end{array}
              \right]$&--------------- $\left[ \begin{array}{c}
                + \mbox{hi} \\ 
                - \mbox{lo} \\
                - \mbox{bk} \\
                 + \mbox{rd} \\
                +  \mbox{ATR} \\
                \end{array}
              \right]$&--------------- $\left[ \begin{array}{c}
                + \mbox{hi} \\ 
                - \mbox{lo} \\
                 + \mbox{bk} \\
                  + \mbox{rd} \\
                +  \mbox{ATR} \\
                \end{array}
              \right]$& ---------------  $\left[ \begin{array}{c}
                + \mbox{hi} \\ 
                - \mbox{lo} \\
                - \mbox{bk} \\
                + \mbox{rd} \\
                +  \mbox{ATR} \\
                \end{array}
              \right]$  \\ \hline 
${\cal R}_2$ &N.A.&N.A.&N.A.&N.A. \\ \hline 
${\cal R}_3$ &N.A.&N.A.&N.A.&N.A. \\ \hline
${\cal R}_4$ &N.A.&N.A.&N.A.&N.A. \\ \hline 
%${\cal R}_passe$ &N.A.&N.A.&N.A.&N.A. \\ \hline 
${\cal R}_5$ &N.A.&N.A.&N.A.&N.A. \\ \hline
${\cal R}_6$ &N.A.&N.A.&N.A.&N.A. \\ \hline
${\cal R}_7$ &N.A.&N.A.&N.A.&N.A. \\ \hline 
Out.   &{\em \'{a}-u}&{\em e-\"{u}}&{\em o-u}&{\em \"{o}-\"{u}} \\ \hline \hline
\end{tabular}
\normalsize

%\begin{itemize}
%\item ${\cal R}_2$ can't apply (by UIP)---everything is already [+rd].
%\item ${\cal R}_4$ can't apply---all the vowels are [+hi].
%\item ${\cal R}_passe$  can't apply (by UIP)---everything is already specified for [bk].
%\item ${\cal R}_5$  can't apply (by UIP)---everything is already specified for [rd]. 
%\item ${\cal R}_6$  can't apply---all the vowels are [+hi].
%\item ${\cal R}_7$   can't apply (by UIP)---everything is already specified for [bk].

%\end{itemize}





\vspace{.4in}



%\clearpage


\begin{example}\label{longo}Generating the {\em \'{o}/\H o} alternation
 
\end{example}
\footnotesize
\begin{tabular}[l]{||c| c| c| c| c||} \hline \hline 
 & \'{a} {\underline{\hspace{.1in}}} & e {\underline{\hspace{.1in}}} & o {\underline{\hspace{.1in}}} & \"{o} {\underline{\hspace{.1in}}} \\ 
In.  &$\left[ \begin{array}{c}
                - \mbox{hi} \\ 
                + \mbox{lo} \\
                + \mbox{bk} \\
               -  \mbox{rd} \\
               + \mbox{ATR} \\
                \end{array}
                \right]$ 
 $\left[ \begin{array}{c}
                - \mbox{hi} \\ 
                - \mbox{lo} \\
                + \mbox{rd} \\
                +  \mbox{ATR} \\
                \end{array}
              \right]$  &
                $\left[ \begin{array}{c}
                - \mbox{hi} \\ 
                - \mbox{lo} \\
                - \mbox{bk} \\
               -  \mbox{rd} \\
               -  \mbox{ATR} \\
                \end{array}
                \right]$  
 $\left[ \begin{array}{c}
                   - \mbox{hi} \\ 
                - \mbox{lo} \\
                + \mbox{rd} \\
                +  \mbox{ATR} \\
                \end{array}
              \right]$  &
                $\left[ \begin{array}{c}
                - \mbox{hi} \\ 
                - \mbox{lo} \\
                + \mbox{bk} \\
               +  \mbox{rd} \\
               +  \mbox{ATR} \\
                \end{array}
                \right]$ 
$\left[ \begin{array}{c}
                 -  \mbox{hi} \\ 
                - \mbox{lo} \\
                + \mbox{rd} \\
                +  \mbox{ATR} \\
                \end{array}
              \right]$ &
               $\left[ \begin{array}{c}
                - \mbox{hi} \\ 
                - \mbox{lo} \\
                - \mbox{bk} \\
               +  \mbox{rd} \\
               +  \mbox{ATR} \\
                \end{array}
                \right]$
$\left[ \begin{array}{c}
                - \mbox{hi} \\ 
                - \mbox{lo} \\
                + \mbox{rd} \\
                +  \mbox{ATR} \\
                \end{array}
                \right]$   \\ \hline 
${\cal R}_1$ &--------------- $\left[ \begin{array}{c}
                - \mbox{hi} \\ 
                - \mbox{lo} \\
                + \mbox{bk} \\
                 + \mbox{rd} \\
                +  \mbox{ATR} \\
                \end{array}
              \right]$&--------------- $\left[ \begin{array}{c}
                - \mbox{hi} \\ 
                - \mbox{lo} \\
                - \mbox{bk} \\
                 + \mbox{rd} \\
                +  \mbox{ATR} \\
                \end{array}
              \right]$&--------------- $\left[ \begin{array}{c}
                - \mbox{hi} \\ 
                - \mbox{lo} \\
                 + \mbox{bk} \\
                  + \mbox{rd} \\
                +  \mbox{ATR} \\
                \end{array}
              \right]$& ---------------  $\left[ \begin{array}{c}
                - \mbox{hi} \\ 
                - \mbox{lo} \\
                - \mbox{bk} \\
                + \mbox{rd} \\
                +  \mbox{ATR} \\
                \end{array}
              \right]$  \\ \hline 
${\cal R}_2$ &N.A.&N.A.&N.A.&N.A. \\ \hline 
${\cal R}_3$ &N.A.&N.A.&N.A.&N.A. \\ \hline

${\cal R}_4$ &N.A.&N.A.&N.A.&N.A. \\ \hline 
%${\cal R}_passe$ &N.A.&N.A.&N.A.&N.A. \\ \hline 
${\cal R}_5$ &N.A.&N.A.&N.A.&N.A. \\ \hline
${\cal R}_6$ &N.A.&N.A.&N.A.&N.A. \\ \hline
${\cal R}_7$ &N.A.&N.A.&N.A.&N.A. \\ \hline 
Out.  &{\em \'{a}-\'{o}}&{\em e-\H o}&{\em o-\'{o}}&{\em \"{o}-\H o} \\ \hline  \hline 
\end{tabular}


\normalsize


%\clearpage
\vspace{.4in}
In (\ref{eoO}) rules ${\cal R}_1$ and  ${\cal R}_6$ affect all forms.  ${\cal R}_2$ only affects forms triggered by back vowels, and   ${\cal R}_5$ only affects  forms triggered by front vowels.   The input vowel is merely specified [-hi, -lo], but a three way  {\em e/\"{o}/o} alternation is generated.

\clearpage


\begin{example}\label{eoO}Generating the {\em e/\"{o}/o} alternation
\end{example}
\footnotesize
\begin{tabular}[l]{||c| c| c| c| c||} \hline \hline 
 & \'{a} {\underline{\hspace{.1in}}} & e {\underline{\hspace{.1in}}} & o {\underline{\hspace{.1in}}} & \"{o} {\underline{\hspace{.1in}}} \\ 
In.  &$\left[ \begin{array}{c}
                - \mbox{hi} \\ 
                + \mbox{lo} \\
                + \mbox{bk} \\
               -  \mbox{rd} \\
               +  \mbox{ATR} \\
                \end{array}
                \right]$ 
 $\left[ \begin{array}{c}
                 - \mbox{hi} \\ 
                - \mbox{lo} \\
              %  +  \mbox{ATR} \\
                \end{array}
              \right]$  &
                $\left[ \begin{array}{c}
                - \mbox{hi} \\ 
                - \mbox{lo} \\
                - \mbox{bk} \\
               -  \mbox{rd} \\
               -  \mbox{ATR} \\
                \end{array}
                \right]$  
 $\left[ \begin{array}{c}
                  - \mbox{hi} \\ 
                - \mbox{lo} \\
          %      +  \mbox{ATR} \\
                \end{array}
              \right]$  &
                $\left[ \begin{array}{c}
                - \mbox{hi} \\ 
                - \mbox{lo} \\
                + \mbox{bk} \\
               +  \mbox{rd} \\
              +  \mbox{ATR} \\
                \end{array}
                \right]$ 
$\left[ \begin{array}{c}
                - \mbox{hi} \\ 
                - \mbox{lo} \\
           %     +  \mbox{ATR} \\
                \end{array}
              \right]$ &
               $\left[ \begin{array}{c}
                 - \mbox{hi} \\ 
                - \mbox{lo} \\
                - \mbox{bk} \\
               +  \mbox{rd} \\
              +  \mbox{ATR} \\ 
                \end{array}
                \right]$
$\left[ \begin{array}{c}
                - \mbox{hi} \\ 
                - \mbox{lo} \\
             %   +  \mbox{ATR} \\
                \end{array}
                \right]$   \\ \hline 
${\cal R}_1$ &--------------- $\left[ \begin{array}{c}
                - \mbox{hi} \\ 
                - \mbox{lo} \\
                + \mbox{bk} \\
                
             %   +  \mbox{ATR} \\
                \end{array}
              \right]$&--------------- $\left[ \begin{array}{c}
                - \mbox{hi} \\ 
                - \mbox{lo} \\
                - \mbox{bk} \\
                
            %    +  \mbox{ATR} \\
                \end{array}
              \right]$&--------------- $\left[ \begin{array}{c}
                - \mbox{hi} \\ 
                - \mbox{lo} \\
                 + \mbox{bk} \\
                
           %     +  \mbox{ATR} \\
                \end{array}
              \right]$& ---------------  $\left[ \begin{array}{c}
                - \mbox{hi} \\ 
                - \mbox{lo} \\
                - \mbox{bk} \\
                
            %    +  \mbox{ATR} \\
                \end{array}
              \right]$ \\ \hline 
${\cal R}_2$ & ---------------  $\left[ \begin{array}{c}
                - \mbox{hi} \\ 
                - \mbox{lo} \\
                + \mbox{bk} \\
                + \mbox{rd} \\
          %      +  \mbox{ATR} \\
                \end{array}
              \right]$&N.A.& ---------------  $\left[ \begin{array}{c}
                - \mbox{hi} \\ 
                - \mbox{lo} \\
                + \mbox{bk} \\
                + \mbox{rd} \\
           %     +  \mbox{ATR} \\
                \end{array}
              \right]$& N.A. \\ \hline 
${\cal R}_3$ &N.A.&N.A.&N.A.&N.A. \\ \hline
${\cal R}_4$ &N.A.&N.A.&N.A.&N.A. \\ \hline
${\cal R}_5$ &N.A.&--------------- $\left[ \begin{array}{c}
                - \mbox{hi} \\ 
                - \mbox{lo} \\
                - \mbox{bk} \\
                 -  \mbox{rd} \\
            %    +  \mbox{ATR} \\
                \end{array}
              \right]$&N.A.& ---------------  $\left[ \begin{array}{c}
                - \mbox{hi} \\ 
                - \mbox{lo} \\
                - \mbox{bk} \\
                + \mbox{rd} \\
            %    +  \mbox{ATR} \\
                \end{array}
              \right]$ \\ \hline 
${\cal R}_6$ &---------------  $\left[ \begin{array}{c}
                - \mbox{hi} \\ 
                - \mbox{lo} \\
                + \mbox{bk} \\
                + \mbox{rd} \\
                +  \mbox{ATR} \\
                \end{array}
              \right]$&  ---------------  $\left[ \begin{array}{c}
                - \mbox{hi} \\ 
                - \mbox{lo} \\
                - \mbox{bk} \\
                - \mbox{rd} \\
                -  \mbox{ATR} \\
                \end{array}
              \right]$  &---------------  $\left[ \begin{array}{c}
                - \mbox{hi} \\ 
                - \mbox{lo} \\
                + \mbox{bk} \\
                + \mbox{rd} \\
                +  \mbox{ATR} \\
                \end{array}
              \right]$&---------------  $\left[ \begin{array}{c}
                - \mbox{hi} \\ 
                - \mbox{lo} \\
                - \mbox{bk} \\
                + \mbox{rd} \\
                +  \mbox{ATR} \\
                \end{array}
              \right]$ \\ \hline 
${\cal R}_7$ &N.A.&N.A.&N.A.&N.A. \\ \hline 
Out.  &{\em \'{a}-o}&{\em e-e}&{\em o-o}&{\em \"{o}-\"{o}} \\ \hline  \hline
\end{tabular}

\normalsize
%\clearpage
\vspace{.4in}
We thus see that a complex set of vowel-vowel interactions can be 
derived using a set of simple rules, appropriately applied in 
accordance with the UIP. As mentioned in passing, there clearly remain other problems to solve in the domain of Hungarian vowels, and it will be interesting to see if  the present proposal can shed new light on these 
classic problems. 
While there is no logical necessity to  derive these
patterns exclusively with the  equivalent of feature-filling rules, doing so 
allows us to simplify the account of vowel harmony in a 
straightforward fashion, a topic we now turn to.


\section{Mixed stems}

By generating 
 Hungarian vowel harmony in this way, we find it is unnecessary explain 
why harmony does not occur between stems in a compound (e.g., {\it balta+ny\'el} `hatchet handle') or between a prefix and a
 stem (e.g., {\it meg+l\'at} `catch sight of'), despite the standard view on this matter as expressed by authors like 
Spencer (1996:178) and  Sipt\'ar  \&  T\"orkenczy  (2000: Chapter 6).  These authors
attribute the failure of harmony to apply in such cases to the assignment 
of prefixes and first compound members to separate phonological words from 
the stem that follows them.
 While such a division  may be reasonable on other grounds, vowel harmony, 
or rather its failure to apply, provides no evidence for a phonological word break 
between such constituents. Under the model presented here, all prefix and stem 
vowels are fully specified\footnote{Aside from some  epenthetic, `unstable' vowels, 
which receive their full specification from within their stem.} and the 
vowel harmony rules I propose 
only affect underspecified vowels, when interpreted using the UIP. 







\section{Conclusions}

We have seen that the goal of conciseness, 
or maximal generality, in rule formulation was 
inconsistent with the search for UG. Rules were formulated in order to generate a given corpus of data, not with the aim of modeling a single human 
phonological component. It is desirable to have a single 
interpretive procedure for determining whether a representation satisfies the 
Structural Description of a rule, and thus serves as an input to the rule.
 This single procedure 
should work for all rules in all languages. In this paper, I have explored a logical alternative to the interpretive procedure adopted in $SPE$, and argued that this Unified Interpretive Procedure 
solves longstanding problems by making it unnecessary to refer to a third value of binary features
[$\emptyset$F], to introduce negation into lexical representations ({\em e.g.} [{\sc not}+rd]) or  to introduce a {\sc Feature-filling/Feature-changing} diacritic on rules.








I have argued  that a small set of rules, provided with the UIP can generate a 
set of data as complex as the target-trigger relations of Hungarian vowel harmony.
The UIP provides us with a principled distinction between feature filling
 and feature changing rules, one that is read off of the rule's representation
 mechanically.  We do not need  labels (or a clever homunculus) to tell us if a
rule is feature-filling or feature changing, and we do not need diacritics in 
representations to refer to features that are not present. If the purely feature filling analysis 
of Hungarian presented here\footnote{Or another feature filling analysis.} is in fact valid, 
then we have shown that this conclusion leads to an overall simplification in our understanding of Hungarian 
phonology---there is less evidence for positing phonological domains within words to account for the 
lack of harmony in  mixed stems and compounds. Not surprisingly, rather 
than being ``ad-hoc and sterile'', attention 
to theoretical detail can lead to improved analyses.





One might wonder whether the UIP is just a notational variant of the rule 
labels `feature-filling' and `feature changing', since it appears to replace these two labels with a disjunctive condition on rule application. There is a good argument that 
it is not. Labels could be used with great latitude, in principle. For example, using the mechanism  of rule labeling,
 a rule could be stated that filled in values for [rd] only on segments 
{\em lacking} specification for [hi]. The notion of a rule label does not preclude this 
possibility, though I believe it is not attested, and the UIP tells us why it is not---feature-filling and feature-changing differences derive from the UIP. The UIP
uses the $SC$ of a rule to determine which features must be absent from a representation for it to serve as an input to that same rule. In other words, the UIP explains why just the feature that appears in the $SC$ is the one 
for which a rule can require obligatory absence. 


The UIP 
allows representations in the lexicon to be non-distinct, and it even makes it possible for one representation 
to subsume another, since  the UIP will allow the two 
 to be treated differently by the 
grammar in a principled fashion. Specifically, a rule can affect a 
representation $A$,  yet fail to affect another representation $B$, 
even when $A$
 is more general than (subsumes)  $B$. As the reference to earlier work by 
Inkelas \& Orgun, Kiparsky, Archangeli and McCarthy 
demonstrates, this has been a longstanding problem in phonological theory. 

By developing a workable (albeit incomplete)
 underspecification analysis for Hungarian, we are working towards a simpler 
model of Hungarian phonology, in general. If the 
contrast between suffix alternations and prefix constancy, 
for example, does not call for a stratal phonology, the question arises of how 
general this result is---both for Hungarian and for universal 
phonology. Can the acceptance of the UIP as formulated here 
be used to simplify other analyses that drew on the arsenal of traditional generative phonology, such as the Derived Environment Condition, the 
Strict Cycle Condition, Structure
Preservation and the  Elsewhere�Condition? This important issue will provide
fertile terrain for future research. 



At this point we can compare the quotation from Morris Halle 
with which we began  to a competing view of the 
 proper goals of phonology:

\begin{quotation}


\noindent We urge a reassessment of this essentially formalist position. If phonology is separated from the principles of well-formedness (the `laws') that drive it, the resulting loss of constraint and theoretical depth will mark a major defeat for the enterprise
[Prince \& Smolensky (1993: 198, see also p.3)]
\end{quotation}

\noindent This paper is an attempt to support Halle's view. We should develop the necessary formal tools to 
express phonological processes, and we should understand how our notation works, 
whether it is  constraint- or rule-based. I find this to be a more interesting and 
pressing problem, one that can lead to greater `theoretical depth', than a taxonomic 
encoding of 
dubious `principles of wellformedness'.



\vspace{.2in}



\noindent{{\bfseries References}}
\small
\begin{reflist}

 Anderson, Stephen. 1985. {\em Phonology in the Twentieth Century}. University of Chicago.


Archangeli, Diana. 1988. {\em Underspecification in Yawelmani Phonology and Morphology}. (revision of 1984 MIT dissertation). New York: Garland. 


Bayer, Sam \&  Mark Johnson. 1995. Features and Agreement. {\em Proceedings of the 33rd Annual Meeting of the Association for Computational Linguistics.} San Francisco: Morgan Kaufmann.

Chomsky, N. 1986. {\it Knowledge of Language.} Westport, CT: Praeger.



Chomsky, Noam. 1967. Some General Properties of Phonological Rules. {\em Language} 43:102-128.

Chomsky, Noam and Morris Halle. 1968. {\em The Sound Pattern of English}. New York: 
Harper and Row.

Hale, Mark \&  Charles Reiss. 2000. The subset principle in phonology: Why the {\em tabula} can't be {\em rasa}. Under revision for  {\em Journal of Linguistics}. 




Hale, M. \& C. Reiss. 1998. Formal and empirical arguments concerning phonological acquisition. {\it Linguistic Inquiry} 29: 656-683.


Halle, Morris. 1975.  Confessio Grammatici. {\em Language} 51:525-35.


     



Halle, Morris \& G.N. Clements. {\em Problem Book in Phonology}. Cambridge, MA: MIT Press.

Inkelas, Sharon. 1996. Archiphonemic Underspecification. Ms. UC Berkeley. 

Inkelas, Sharon and Orhan Orgun. 1995. Level ordering and economy in the lexical phonology of Turkish. {\em Language 71}. 763-793.

%Keresztes, K. 2000. {\em Hungarolingua Grammatica: \\ A Practical Hungarian Grammar}. http://www.nyariegyetem.hu/en/grammatica.html.


Kiparsky, Paul. 1985. Some Consequences of Lexical Phonology.  {\em Phonology Yearbook} 2:85-138.

McCarthy, John. 1994.  The phonetics and phonology of Semitic pharyngeals. In P. Keating, ed., {\em Phonological Structure and Phonetic Form: Papers in Laboratory Phonology III}. 191-233. Cambridge University Press.


Prince, Alan and Paul Smolensky. 1993. {\em Optimality Theory: Constraint Interaction in Generative Grammar}. 
Technical Report Rutgers Center for Cognitive Science, Rutgers University, New Brunswick, N.J.

Reiss, Charles. 2003a. Towards a theory of fundamental phonological relations. To appear in A.M.
di Sciullo, ed.,  {\it Papers on  Linguistic  Asymmetry.}  John Benjamins Press.. 

Reiss, Charles. 2003b. Quantification in phonological rules. To appear in {\it The Linguistic Review 20: Special Issue on Typology in Phonology}, Spring 2003.


Ringen, Catherine  \& Robert  Vago. 1998. Hungarian vowel harmony in Optimality Theory. {\it Phonology} 15.3: 393-416.

Sipt\'ar, P\'eter. 1994. The vowel inventory of Hungarian: Its size and structure. {\it The Even Yearbook 1994}:175--184. Budapest: ELTE SEAS Working Papers in Linguistics.


Sipt\'ar, P\'eter \&  Mikl\'os T\"orkenczy. 2000. {\em The Phonology of Hungarian.}
Oxford: Oxford University Press.

Spencer, Andrew. 1996.  {\em Phonology}. Oxford: Blackwell.



\end{reflist}


\end{document}



   \begingroup
     \parindent 0pt
     \parskip 2ex
     \def\enotesize{\normalsize}
     \theendnotes
     \endgroup


\end{document}






















To do: 

nnnnnnnnnnnnnnnnnnnnnnnnnnnnnnnnnnnnnnnnnnnnnnnnnn

3explain why I don't assume underlying values for the Hungarian vowels.
4Change Russian to Polish or something similar.




 This is illustrated schematically using $\Sigma$ to represent a partial feature bundle (everything in the $SD$ except
 a value for F)
 and F as one of the  features belonging to the same representation.
  

  
\begin{example}\label{alpharules} Unified  Interpretive Procedure and $\alpha$-notation
  
\noindent Applies to 

\begin{itemize}
\item[a.]{{$[\Sigma$, $-\alpha$F$]$} $\rightarrow$  {$[\alpha$F$]$}  in $Env$  \hspace{.3in} (applies to $Q$ if $[-\alpha$F$] \in$  $Q$) \hspace{1in}  {\sc or}}
\item[b.]{$[\Sigma]$ $\rightarrow$  {$[\alpha$F$]$}   in $Env$  \hspace{.3in}   (applies to $Q$ if [$-\alpha$F$] \notin$  $Q$ and  $[-\alpha$F$] \notin$  $SD$)  }
\end{itemize}
\end{example}

\noindent Both (\ref{alpharules}ab) are possible, as long as (b) is interpreted so that a 
representation that is subsumed by $\Sigma$, but specified for any value of F, does not constitute an input to the rule.



 After some general remarks, I will demonstrate that merely replacing the $SPE$ interpretive procedure with one incorporating the notion of subsumption runs into other problems, also arising from the use of underspecification. I will then propose a solution to this problem in the form of a unified interpretive procedure.




We can generalize this result in terms of subsumption.
 First consider a schematic example:

\begin{example}
\begin{tabular}{l|lll} 
&A&B&C \\ \hline
F&&+&-\\
G&+&+&+\\
H&-&-&-\\
\end{tabular}
\end{example}
 
\noindent If we let F stand for [voiced], then A corresponds to the Turkish /D/, B corresponds to /d/ and C corresponds to /t/. We can now generalize the treatment of input representations given our UIP:

\begin{example} 
Given three representations A, B, C, where  A is not specified for F and A 
 subsumes  B, 
and A subsumes  C, but 
B is distinct from C with respect to  feature F (that 
is, A subsumes both B and C, but neither subsumes the 
other) then there is no SD that refers to F that A does 
not satisfy, but B and C do.
\end{example}

%\clearpage



\begin{example}\label{longu}Generating the {\em \'u/\H u} alternation
  
\end{example}
\footnotesize
\begin{tabular}[l]{||c| c| c| c| c||} \hline \hline 
 & \'{a} {\underline{\hspace{.1in}}} & e {\underline{\hspace{.1in}}} & o {\underline{\hspace{.1in}}} & \"{o} {\underline{\hspace{.1in}}} \\ 
In.  &$\left[ \begin{array}{c}
                - \mbox{hi} \\ 
                + \mbox{lo} \\
                + \mbox{bk} \\
               -  \mbox{rd} \\
               +  \mbox{ATR} \\
                \end{array}
                \right]$ 
 $\left[ \begin{array}{c}
                + \mbox{hi} \\ 
                - \mbox{lo} \\
                + \mbox{rd} \\
                +  \mbox{ATR} \\
                \end{array}
              \right]$  &
                $\left[ \begin{array}{c}
                - \mbox{hi} \\ 
                - \mbox{lo} \\
                - \mbox{bk} \\
               -  \mbox{rd} \\
               -  \mbox{ATR} \\
                \end{array}
                \right]$  
 $\left[ \begin{array}{c}
                   + \mbox{hi} \\ 
                - \mbox{lo} \\
                + \mbox{rd} \\
                +  \mbox{ATR} \\
                \end{array}
              \right]$  &
                $\left[ \begin{array}{c}
                - \mbox{hi} \\ 
                - \mbox{lo} \\
                + \mbox{bk} \\
               +  \mbox{rd} \\
               +  \mbox{ATR} \\
                \end{array}
                \right]$ 
$\left[ \begin{array}{c}
                 + \mbox{hi} \\ 
                - \mbox{lo} \\
                + \mbox{rd} \\
                +  \mbox{ATR} \\
                \end{array}
              \right]$ &
               $\left[ \begin{array}{c}
                - \mbox{hi} \\ 
                - \mbox{lo} \\
                - \mbox{bk} \\
               +  \mbox{rd} \\
               +  \mbox{ATR} \\
                \end{array}
                \right]$
$\left[ \begin{array}{c}
                + \mbox{hi} \\ 
                - \mbox{lo} \\
                + \mbox{rd} \\
                +  \mbox{ATR} \\
                \end{array}
                \right]$   \\ \hline 
${\cal R}_1$ &--------------- $\left[ \begin{array}{c}
                + \mbox{hi} \\ 
                - \mbox{lo} \\
                + \mbox{bk} \\
                 + \mbox{rd} \\
                +  \mbox{ATR} \\
                \end{array}
              \right]$&--------------- $\left[ \begin{array}{c}
                + \mbox{hi} \\ 
                - \mbox{lo} \\
                - \mbox{bk} \\
                 + \mbox{rd} \\
                +  \mbox{ATR} \\
                \end{array}
              \right]$&--------------- $\left[ \begin{array}{c}
                + \mbox{hi} \\ 
                - \mbox{lo} \\
                 + \mbox{bk} \\
                  + \mbox{rd} \\
                +  \mbox{ATR} \\
                \end{array}
              \right]$& ---------------  $\left[ \begin{array}{c}
                + \mbox{hi} \\ 
                - \mbox{lo} \\
                - \mbox{bk} \\
                + \mbox{rd} \\
                +  \mbox{ATR} \\
                \end{array}
              \right]$  \\ \hline 
${\cal R}_2$ &N.A.&N.A.&N.A.&N.A. \\ \hline 
${\cal R}_3$ &N.A.&N.A.&N.A.&N.A. \\ \hline
${\cal R}_4$ &N.A.&N.A.&N.A.&N.A. \\ \hline 
%${\cal R}_passe$ &N.A.&N.A.&N.A.&N.A. \\ \hline 
${\cal R}_5$ &N.A.&N.A.&N.A.&N.A. \\ \hline
${\cal R}_6$ &N.A.&N.A.&N.A.&N.A. \\ \hline
${\cal R}_7$ &N.A.&N.A.&N.A.&N.A. \\ \hline 
Out.   &{\em \'{a}-u}&{\em e-\"{u}}&{\em o-u}&{\em \"{o}-\"{u}} \\ \hline \hline
\end{tabular}
\normalsize


\footnote{A reviewer suggests the possibility that German is not the best choice to illustrate this process since it has been argued that the voiceless obstruents in German are, in fact, unspecified for voicing underlyingly. We could replace German by Polish or Russian or any other of the many attested cases of coda devoicing involving a neutralization of a two-way underlying distinction.}
