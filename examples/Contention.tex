

\documentclass[11pt]{article}

\setlength{\oddsidemargin}{0in}
\setlength{\evensidemargin}{0in}
\setlength{\topmargin}{0.0in}
\setlength{\headheight}{0in}
\setlength{\headsep}{0.5in}
\setlength{\textwidth}{6.5in}
\setlength{\textheight}{8.5in}
\setlength{\parindent}{0in}
\setlength{\parskip}{0.05in}

\usepackage{times}
\usepackage{psfig}
\usepackage{fancyheadings}

\begin{document}

\pagestyle{fancy}
\cfoot{Contention Protocols: \thepage}

\title{Some Utilization Analyses for ALOHA and CSMA Protocols}

\author{Norman Matloff \\
University of California at Davis\\
        \copyright{}2001, N. Matloff}                      

\date{November 20, 2001}

\maketitle

\section{Contention-Based LANs}

In {\bf contention-based local-area networks} we have all stations on
a common channel, such as a cable or a specific radio frequency, and
the stations contend with each other for access to that channel.  Only
one station can transmit at a time, so if two or more stations attempt
to use the channel at the same time (a {\bf collision}), all of their
frames will be garbled, and some mechanisms must be devised to (a)
detection the problem and (b) to arrange for retransmission.

Here we will be interested in the efficiencies of such networks.

\section{Notation and Assumptions}

Let S denote the mean number of ``original'' frames generated at all
nodes of a contention-based network (ALOHA, CSMA, etc.) per unit 
time.  By ``original'' we mean that we are not including retransmitted
frames.  Then let G denote the mean number of all frames, original
or retries, generated per unit time.

Scale time so that a frame takes one unit of time to transmit.  Then
since we can transmit at most one frame per unit of time, we must have
$S \leq 1$.  Accordingly, S is the utilization of the channel, that
is the proportion of the time that the channel is carrying data which
reach their destination intact.  (Think about what happens over a time
interval of, say, length 10000.  The channel will be busy sending frames
(not counting retransmits) for $10000 \times S \times 1$ of this time
10000, thus S proportion of that time.)

We will be interested in deriving S as a function of G, and in
acquiring some qualitative insight from that function.

Note that $S = G P_0$, where $P_0$ is the probability that no other
frame will be transmitted during the time a given frame, which we
will call the {\bf reference frame}, is being sent.  (Of course,
this includes the case in which a frame overlaps with the reference
frame by even a small amount; this still is a collision, and thus
still requires retransmission.)   So, our task now is to express
$P_0$ in terms of G.

We will assume, as is common in such analyses, that the number of
frames transmitted has a Poisson distribution:  The probability
that k frames are transmitted during interval of length s is

\begin{equation}
\frac{{(Gs)}^k e^{-Gs}}{k!}, k = 0,1,2,...
\end{equation}

This also means that the times between successive frames has an
exponential distribution with mean 1/G.  (It can be proven that this
is a property of Poisson processes---event counts have a Poisson
distribution if and only if the interevent times are exponentially
distributed.)

\section{ALOHA}

ALOHA was developed at the University of Hawaii by Norman Abramson and
others.  It consisted of a radio link between stations on several
islands.

The protocol was simple:  A station would transmit whenever it had data
to send.  If it were unlucky enough that some other station has data to
send around that time (whether earlier or later), and the transmission
time intervals of the stations overlap, then of course the destination
station would never receive it properly, and thus never send an ACK.
The sending station would then timeout, and retransmit.

\subsection{Utilization Analysis:  Ordinary ALOHA} 

Let $t_0$ denote the time our reference frame begins transmission.  The
transmission will be successful if and only if there are no other
transmissions which begin during $(t_0 - 1,t_0 + 1]$.  Setting s = 2
(since the interval length is 2) and k = 0 in our Poisson
formula,\footnote{You may wonder why we do not set k = 1.  The point is
that we are \underline{given} that there is a transmission at $t_0$, so
we are actually interested in the \underline{conditional} distribution
of the times of other frames sent, given a transmission at $t_0$.  But
those other frames still follow the Poisson process described above.
This is a subtle point, whose careful derivation would be well beyond
the scope of our course, but can be at least seen in outline form by
recalling that the times between successive frames has an exponential
distribution.  Since the exponential distribution is ``memoryless,'' the
chances of having a frame in, say, $(t_0,t_0+1)$ is the same, whether or
not we have the knowledge that a frame transmission began at $t_0$.} we
have $P_0 = e^{-2G}$.  So, 

\begin{equation}
S = G e^{-2G}.
\end{equation}

Setting $\frac{dS}{dG} = 0$, we find that S is maximum when G = 0.5,
and that at that time $S = \frac{1}{2e}$.  (Note that G is not
under our control, so the specific value of G which maximizes S is
not so important.  We are simply interested in knowing how large
S can get, in this case 1/(2e).)

In other words, the best-case utilization of ALOHA is about 18\%,
rather poor.  This of course was the later motivation for ALOHA
refinements and other methods.

\subsection{Utilization Analysis:  Slotted ALOHA}

If you take another look at the analysis of ALOHA above, it is clear
that ALOHA suffers from the very wide ``window of vulnerability''
$(t_0 - 1,t_0 + 1]$.  The whole idea of slotted ALOHA is to make
this window narrower, thus improving performance.

Here transmissions are allowed to occur only at integer time points, 1,
2, 3, ...  A frame will be transmitted at $t_0$ if and only if it became
``ready'' during $(t_0 - 1,t_0]$ (either it originated during that time,
or became ready for retransmission after failing in the past).  The same
analysis as we have above then yields

\begin{equation}
S = G e^{-G}.
\end{equation}

This gives a maximum utilization of 1/e, double that of nonslotted
ALOHA, about 37\%.

\subsection{Comparison}

The relations between S and G for ALOHA and slotted ALOHA are plotted
in this figure:

\vspace{0.2in}
\par
\psfig{file=SVsG.PS}
\par

Note that both curves fall off quickly once we pass their peaks.  The
more load we place on the channel, the less actually gets through.
However, slotted ALOHA is clearly the superior performer.

\section{CSMA}

The Carrier Sense Multiple Access (CSMA) protocol, or more precisely its
refinement, Carrier Sense Multiple Access/Collision Detect (CSMA/CD),
was developed by Robert Metcalfe at Xerox Corp. (who later founded
3-Com) and others.  It is commonly known as Ethernet, and is in very
wide usage.  When it first became popular, it used as its medium
coaxial cable, though these days more complex arrangements are common.

The idea of CSMA is ``listen before talk.''  Before attempting to
transmit, a station, say X, will sense the cable for a carrier signal,
which, if present, signifies that some other station, say Y, is sending.
In such a case, station X reschedules a future retransmission, by
generating a random wait time.

This is called {\bf nonpersistent} CSMA.  A variation is {\bf
1-persistent} CSMA, in which station X continues to listen to the line,
and then sends immediately after Y finishes.  The problem with
1-persistent CSMA is that a third station, say Z, also wants to send
while Y is sending; in this case, X and Z would collide right after Y is
done.  

Thus a common variation is {\bf p-persistent} CSMA, in which stations X
and Z would generate random numbers, and transmit right after Y is done
only with probability p.  If the random number tells a station to wait,
it waits---``backs off''---for random amount of time and then tries again.
The advantage is that the probability X and Z collide in the above
scenario is only $p^2$.  On the other hand, if, say, only Z had been
waiting but not X, then Z might back off unnecessarily.


Even if a station with a message to send checks the line and ``hears''
nothing, a collision may still occur, because another station may be
currently transmitting but due to propagation delay its frame may not
have arrived yet at the first station.  For that reason, the CD
mechanism was added to CSMA: While a station is sending, it continues to
monitor the line, to check whether its frame is on the line intact.  If
a collision occurs, both stations will detect this and cease
transmission.  So, a collision will be detected much earlier than it
would if we merely just wait for an ACK (or lack of one), as in the
ALOHA case.

\subsection{Utilization Analysis:  Nonpersistent CSMA}

(Adapted from {\it Modeling and Analysis of Computer Communications
Networks}, by Jeremiah Hayes, pub. by Plenum, 1984.)

The line will be idle for a while, then busy for a while (whether with
successful transmission or a collision), then idle, then busy, and so
on.  Let B and I denote random variables representing the lengths of the
busy and idle times, respectively.  The mean length of a busy/idle cycle
will be

\begin{equation}
E(B) + E(I)
\end{equation}

During a busy/idle cycle, let T be the time spent successfully
sending a message.  In each busy/idle cycle, there will be either no
successful transmissions (the busy period had a collision) or exactly
one successful transmission.\footnote{This is where we use the fact that
we have nonpersistent CSMA.  With persistent CSMA, a second successful
transmission could follow right on the heels of a previous one, if the
second station sees the first busy and then starts sending right after
the first one finishes.}  By definition of busy period, there will be at
least one station with something to send.  Consider the station which
sends \underline{first}, and let $t_0$ denote the time it starts, so
that it sends during $(t_0,t_0+1)$.  (Note also that $t_0$ is the time
this busy period starts.)  The probability no other station collides
with it is, in the same manner we saw for ALOHA,

\begin{equation}
e^{-G \alpha}
\end{equation}

So, 

\begin{equation}
E(T) = 1 \cdot e^{-G \alpha} + 0 \cdot (1 - e^{-G \alpha}) = e^{-G \alpha}
\end{equation}

The utilization of the line is

\begin{equation}
u = \frac{E(T)}{E(B) + E(I)}
\end{equation}

Let $\alpha$ denote the ratio of end-to-end propagation delay in the
cable to the frame transmission time.  Due to our time scaling, $\alpha$
is also the end-to-end propagation delay.

To derive E(B), note that if there is no collision, then B will equal
the frame transmit time plus the propagation delay, which on average
will be

\begin{equation}
B = 1 + \frac{3}{4} \alpha
\end{equation}

assuming the stations are uniformly spread out along the length of the
cable.  On the other hand, if there is a collision, then B will equal

\begin{equation}
B = 1 + \frac{3}{4} \alpha + D
\end{equation}

where D is defined as follows:  Again some station will be the first to
start, at time $t_0$.  Then D is the amount of time later that the last 
station to send in this busy period starts, that is the last station
starts at time $t_0 + D$.\footnote{Note that here we are using the
fact that this is CSMA, not CSMA/CD.}

We need to find E(B), so we need E(D).  To this end, note that we must
have $D \le \alpha$; recall that this is how the collisions occur in the
first place---a station thinks the line is free but has not received a
transmission in progress yet, due to propagation delay.  Now draw a
number line showing $t_0$, $t_0+D$, $t_0+x$ and $t_0+\alpha$, in that
order, and you will see why $D \le x$ if and only if there are no
transmissions during the time interval $(t_0+x,t_0+\alpha)$.  The
probability of that event is

\begin{equation}
e^{-G(\alpha - x)}
\end{equation}

So, 

\begin{equation}
P(D \le x) = e^{-G(\alpha - x)}
\end{equation}

Thus, the probability density function of D is

\begin{equation}
\frac{d}{dx} P(D \le x) = \frac{d}{dx} e^{-G(\alpha - x)} = 
G e^{-G(\alpha - x)}
\end{equation}

for $0 < x < \alpha$.

Thus

\begin{equation}
E(D) = \int_0^\alpha x \cdot G e^{-G(\alpha - x)} = 
\alpha - \frac{1}{G} (1 - e^{-G\alpha})
\end{equation}

So,

\begin{equation}
E(B) = 1 + 0.75 \alpha + \alpha - \frac{1}{G} (1 - e^{-G\alpha})
\end{equation}

Finally the quantity E(I) is easily determined as follows.  I is the
time until the first transmission following a certain time ($t_0 + 1 +
\alpha$), so it has an exponential distribution with mean 1/G.

Thus,

\begin{equation}
u = \frac{E(T)}{E(B) + E(I)} = 
\frac{Ge^{-G\alpha}}{G(1+1.75\alpha) + e^{-G\alpha}}
\end{equation}

For very small $\alpha$, we have $u \approx G/(G+1)$, suggesting that
CSMA can be very efficient.

\subsection{Refinements of Ethernet}

Originally the 10 megabits per second transmission speed of Ethernet
seemed sufficient.  (The token rings at the time sent at the rate of
4 megabits per second.)  However, with the advent of much faster
protocols such as FDDI, and most importantly the tremendous growth
in network applications and need for speed, the original Ethernet
speed is considered slow today.

A newer technology is Fast Ethernet, which uses a regular CSMA/CD
protocol but increases speed to 100 megabits per second via the
following modifications:

\begin{itemize}
\item Transmission is on three lines instead of one.

\item 8B6T coding is used instead of Machester.

\item The clock rate is 25 MHz instead of 20 MHz.
\end{itemize}

Further refinements have recently led to Gigabit Ethernet, with
speed 1000 megabits per second.

Moreover, instead of the older single-cable Ethernet topology,
Ethernet hubs and switches have become popular.  We will discuss
this in another handout later.  

\section{Simulation Analyses}

Many network protocol analyses are mathematically intractable, so that
simulation must be used instead.  Following is an example of how this
is done.

\begin{verbatim}
  1     
  2     
  3     /*  Sample simulation program:  ALOHA protocol, with a "p-persistent"
  4         feature added.
  5     
  6         There are NNodes network nodes which transmit on the same channel.
  7         Time is slotted, i.e. transmission can begin only at integer times.
  8         If more than one station attempts transmission during a given slot,
  9         they will "collide," corrupting each others' messages, and they
 10         must try again.  To avoid repeated collisions, a station which has
 11         a frame to be transmitted will do so only with probability P.  Its
 12         interface hardware generates a random number between 0 and 1, and
 13         transmits only if the number is less than P; otherwise it waits 
 14         until the next slot and repeats the process.
 15     
 16         We say a station is ACTIVE if it has something to send; otherwise
 17         it is IDLE.  An IDLE station will become active in any given slot
 18         with probability NewMsgProb. 
 19     
 20         We are interested in the long-run average message delay.   Our
 21         approximation to "long-run" will be 10,000 time slots. */
 22     
 23     
 24     #include <stdlib.h>  /* needed for RAND_MAX */
 25     
 26     
 27     #define IDLE 0
 28     #define ACTIVE 1
 29     #define NSLOTS 10000
 30     #define MAXNODES 100
 31     
 32     
 33     int NNodes, /* number of nodes in the network */
 34         State[MAXNODES], /* current node states, IDLE or ACTIVE */
 35         Delay[MAXNODES], /* delays so far to message, if any, at each node *
 36         NMsg, /* number of successfully transmitted messages so far */
 37         SumDelay; /* overall total delays accumulated so far */
 38     
 39     
 40     float P, /* probability that an ACTIVE node will send */
 41           NewMsgProb; /* probability that an IDLE node will become ACTIVE */
 42     
 43     
 44     /* the function Rnd(Prob) simulates a random event of probability
 45        Prob, with the return value 1 meaning the event occurred and 0
 46        meaning that it did not occur */
 47     
 48     int Rnd(Prob)
 49        float Prob;
 50     
 51     {  return((rand() < Prob * RAND_MAX));  }
 52     
 53     
 54     Init(argc,argv)
 55        int argc; char **argv;
 56     
 57     {  int Node;
 58        
 59        sscanf(argv[1],"%f",&NewMsgProb);
 60        sscanf(argv[2],"%f",&P);
 61        sscanf(argv[3],"%d",&NNodes);
 62     
 63        SumDelay = 0;
 64        NMsg = 0;
 65        for (Node = 0; Node < NNodes; Node++)  
 66           State[Node] = IDLE;
 67     }
 68     
 69     
 70     main(argc,argv)
 71        int argc; char **argv;
 72     
 73     {  int Slot,NTry,Node,TryNode;  
 74     
 75        Init(argc,argv);
 76     
 77     
 78        /* simulate the system for NSLOTS time periods */
 79        for (Slot = 0; Slot < NSLOTS; Slot++)  {
 80           NTry = 0;
 81           /* for each node, check whether it has changed from IDLE to
 82              ACTIVE, and if ACTIVE (from just now or beforehand) check
 83              whether it will attempt to transmit */
 84           for (Node = 0; Node < NNodes; Node++)  {
 85              if (State[Node] == IDLE && Rnd(NewMsgProb))  {
 86                 State[Node] = ACTIVE;
 87                 Delay[Node] = 0;
 88              }
 89              if (State[Node] == ACTIVE)  { 
 90                 Delay[Node]++;
 91                 /* decide whether to transmit */
 92                 if (Rnd(P))  {
 93                    NTry++;
 94                    TryNode = Node;
 95                 }
 96              }
 97           }
 98           /* a successful transmission will occur if exactly one node
 99              attempted transmission */
100           if (NTry == 1)  {
101              NMsg++;
102              SumDelay += Delay[TryNode];
103              State[TryNode] = IDLE;
104           }
105        }
106     
107        if (NMsg > 0)
108           printf("long-run average delay = %f\n",SumDelay/((float) NMsg));
109        else
110           printf("NMsg = 0\n");
111     }
\end{verbatim}


\end{document}

