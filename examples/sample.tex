%\documentclass[preprint]{aastex} 

%Remember that in a tex file, a percent sign means 'comment' and the
%rest of the line will not appear in the printed output.

%another option include preprint2, which gives you two columns. to invoke:
%\documentclass[preprint2]{aastex} 

%Note: If you're doing work for some other class and don't want to use
%aastex formatting, you can use some varient of the following commands,
%which are intrinsic to LATEX. However, some commands below, such as
%\altaffiltext and \deluxetable, are peculiar to aastex.

\documentclass[11pt]{article}
\usepackage[dvips]{graphicx} 

\begin{document}

\title{\LaTeX\ Is Your Friend.}

\author{Nathan Lundblad \altaffilmark{1}\altaffilmark{,2} \\ August 31,1999}
%NOTE THE \\ WHICH SKIPS A LINE
%To use todays date, use the \today command.

\altaffiltext{1}{email: lundblad@UGAstro.berkeley.edu}
\altaffiltext{2}{With \LaTeXe updates and additional commentary by Erik
Shirokoff and Carl Heiles: \protect\today}
\maketitle
\begin{abstract} We present a paper on useful \LaTeX\ stuff.  Make sure
to look at the source code for this document, as that is where the real
story is.  For more fun, look at Leslie Lamport's book in the 705
Campbell bookshelf.  \end{abstract}

\section{The Big Picture}\label{bigpicsec} %NOTE THE LABEL SYNTAX

        \LaTeX\ (pronounced {\it lay}-teck or {\it lah}-teck) is a
designer package based on a typesetting program called \TeX\, which was
originated by Donald Knuth\footnote{The greatest computer scientist in
the world.} of Stanford many many years ago.  \LaTeX\ first appeared in
1985 and is extremely popular, particularly in the scientific community
where it has become an almost universal standard.  Using \LaTeX\ will
result in stunningly beautiful documents and will, in the long run, be
easier to deal with than using Micro\$oft Word$^{tm}$ or
IslandWrite$^{tm}$.  Although creating reports and articles in a
different fashion from what you may be used to can be a little
intimidating at first, a few basic facts and a couple of good sample
documents\footnote{Available all over the place.  You'll get a sample
lab report done in \LaTeX\, for example.  } will take you a long way. 

\section{How To Use \LaTeX: The Most General Possible Summary}
\label{howtosec}

        Remember on PC word processors how there is an option called
{\it reveal codes} or some such? Well, in \LaTeX\ you essentially write
those codes yourself, and then compile them to get your printable
output.  You'll type up these codes in your favorite text editor and
name the file something appropriate with a \verb&.tex& suffix.

        Then you must compile that file at your shell prompt by typing
\verb&latex whatever.tex&.  \LaTeX\ will spit out some random files
(provided you haven't made any errors), including \verb&whatever.dvi&,
which is your printable output.  \LaTeX\ will also print some messages
on your screen.  {\it Be sure to look at these messages!!!!!!!}  If 
your compilation failed, they will attempt to tell you what error you
may have committed.  Once you figure it out, you can Ctrl-C out of the
compilation environment and try again.   The most common error is to
forget the \$ sign on each side of an equation, or to have unmatched
curly brackets.  The error message gives the line number; the easiest
way to find the offending text is to go to that line number in your
editor. {\it NOTE} that many times the error occurs {\it before} the
line number given by the \LaTeX\ output.

        To view the \verb&whatever.dvi& file on your terminal screen,
type \verb&xdvi whatever.dvi& at your shell prompt.  Look it over
carefully and make any changes {\it before printing it on
paper}---support environmentalism! Finally, to print the output when
you're all done, type \verb&dvips whatever.dvi | lp&.~\footnote{In this
command, \textbf{dvips} creates a postscript file from
\textbf{whatever.dvi}. The $\mathbf{\vert}$ \textbf{lp}  ``pipes'' this
file to the printer; omitting this part would print the postscript file
onto your screen---something you don't want because it is
uninterpretable.}  It's that simple!


\section{Some Basic Syntax}\label{basicsec}



        Every \LaTeX\ document must be enclosed by a
\verb&\begin{document}& tag and and an \verb&\end{document}& tag. 
Nothing goes after the latter\footnote{Except for comments which you
don't want to be interpreted.}, but some very important stuff goes
before the former, such as documentclass declarations and suchlike,
which you'll learn about in Section~\ref{stylesec}.   
%NOTE THE REFERENCE TO THE SECTION'S LABEL!!  
%THE TILDE (~) PREVENTS LATEX FROM BREAKING THE LINE AT THAT POINT.  
As you may have noticed, \LaTeX\ reserves more than a few characters for
its own nefarious purposes.  Generally, to produce them in your final
document you must invoke the backslash, like so: ``\verb&\$&'', which
results in a final output like so: \$.  The same method applies to other
special characters: \{ \# \} \%\footnote{The percent sign \% is used for
commenting your code, which is very important in, say, C programming but
not too important in \LaTeX.}.  The observant student in the back of the
room may cleverly ask ``So\ldots how do you create a backslash, if
\verb& \\& represents a skipped line?'' Well, you have to use the
\verb&\verb& (verbatim) environment, which is handily revealed in the
source code. 

	The argument of the \verb$\verb$ environment is delimited by two
identicdal characters; above, we used ampersands. You can use a pair of
any normal character as the delimiter. The \verb$\verb$ environment has
an unfortunate peculiarity: you have to put all of its argument on a
single typed line in the \verb$tex$ file. If you want to do a lot of
verbatim stuff, use \verb$\begin{verbatim}$ and \verb$\end{verbatim}$;
these don't require delimiters. \verb$\begin{verbatim}$ has the
unfortunate peculiarity that it skips and starts a new line.

        There are three kinds of hyphens in \LaTeX: -,--, and ---.  The
first is used for intra-word dashes, the second for number ranges
(41--42), and the third for the standard intra-sentence dash---it's my
personal favorite.  In other situations, just use whatever looks the
best. 

        Grouping letters and words is accomplished with the \{ and \}
characters.  Most commands only work on one group at a time, so surround
the parts of your text you want to modify with curly brackets.

        Footnotes are incredibly easy to produce, and are automatically
numbered.\footnote{Like So.  Voil\`{a}!}

\section{Labels}\label{labelsec}

        When you're preparing a \LaTeX\ document, it's good but not
necessary to use the ``\verb&\&label'' command.  The use of labels
ensures that you can refer to sections, equations, figures, and tables
by a name and not a number.  So what's the difference? When you're
inserting, cutting, and pasting, you {\it will} lose count of what
section you're in or what equation is what, which will make referring to
such objects in the text.  Because I labeled the beginning of this
section, I can always refer to it using the label, regardless of whether
I go back and make changes in section order.  For example: ``The current
section is Section~\ref{labelsec}''.  
%NOTE THE REFERENCE 
There are several examples of how labels work in this primer; they are
commented as needed. 

\section{Style files, packages, and user defined commands}\label{stylesec}

        You may have noticed the following line at the beginning of the
source code: \begin{center} {\verb&\documentclass[preprint]{aastex}&}
\end{center} \noindent This line sets a template for the document as a
whole; it tells \LaTeX\ that you want to write an article-type document
using the American Astronomical Society's preprint class \footnote{The
American Mathematical Society and the American Physical Society also
have their own formats.  We like AAS\TeX, and recommend you stick with
it.} package.  Specifically, this command instructs \LaTeX \ to read the
file called \verb$aastex.sty$, which is known as a ``style file''; if
you want to use AAS\TeX \ on your own computer, you need to have this file
available on your path. The AAS\TeX\ class sets the font and layout for
the entire document and it automatically loads some useful packages,
which are collections of new commands that allow you to customize your
document and do nifty things with images and layouts.  


\section{Mathematics}\label{mathsec}

        The great beauty of \LaTeX\ lies in how the math comes out.  It
does numbered equations exceptionally well, enables math within standard
text, and has a shocking number of special characters available. 
Inserting standard equations into a \LaTeX\ document is done with the
\verb&\equation& environment, and works like so:

\begin{equation} \label{laplacian}
\frac{\partial^2 V} {\partial x^2}+\frac{\partial^2 V}
{\partial y^2} + \frac{\partial^2 V}{\partial z^2}=0
\end{equation}

        Laplace would have loved \LaTeX.  You can also do Greek letters
easily:

\begin{equation} \label{gammaeq}
\gamma=\frac{1}{\sqrt{1-\beta^2}}
\end{equation}

\noindent If you want to put mathematics into text, you can use math
mode, which is commonly delimited by dollar signs; thus,
\verb&$\alpha=\beta=\int_0^2 x^{-2.4} dx$& will look like
$\alpha=\beta=\int_0^2 x^{-2.4} dx$.  For an example of how labels work
with equations, look at the code for Equation~\ref{gammaeq}. 

\section{Figures}\label{figsec}

        If you want to bring in plots from IDL or, for that matter, an
arbitrary graphic, you must first make sure that the file in question is
an Encapsulated Postscript File or a Postscript file.  Once you have the
file in the same directory as your \verb&.tex& file, you can insert it
into the document like so:

\begin{figure}[h!]
%the h! tells latex you want the figure inserted here.
\begin{center}
\includegraphics[width=.6\textwidth]{2dgaussian.ps}
\caption{A Gaussian.}\label{gaussfig}
%NOTE THAT LABELS WORK ON FIGURES, TOO!!!
\end{center}
\end{figure}

%In addition to width, you can define height, angle, and scale.  If you
%specify only width or height, the other dimension scales automatically.
%If you specify both, you can stretch the image.  Angle rotates the image
%by some number of degree in the positive direction. Scale multiplies the
%picture's original size by the number you specify.

%When specifying width or height, you must include units.  Some options
%are: \textwidth, in, cm, pt, em, ex. See the Not So Short Guide for more
%info.

If you want to display several pictures together, with size scaling or
stretching or rotation, as in figure \ref{silly}, you can do this.

\begin{figure} \label{silly}
\begin{center}
\includegraphics[width=1in,height=5in]{2dgaussian.ps}
\includegraphics[width=5in,height=1in,angle=180]{2dgaussian.ps}
\includegraphics[scale=0.1,angle=45]{2dgaussian.ps}
\end{center}
\caption{This is a very silly figure!}
\end{figure}

	One of the most difficult tasks for the novice typsetter is
image placement.  \LaTeX places floating bodies where it thinks they
best fit, which isn't always the most logical place in a document. 
There are a number of techniques which can help, however, most placement
problems can be solved simply by making your images smaller. 

\clearpage
%NOTE THAT ONE CAN BREAK THE PAGE BY USING \clearpage....

\section{Tables}\label{tablesec}

        Tables are useful for displaying a large number of results. 
There are two environments provided for tables; \verb&{table}&, which is
a \LaTeX\ resident environment, and \verb&{deluxetable}&, which is an
AAS\TeX\ custom environment.  Examples of both and their use are
provided below. 

\begin{table}[h]
%THE H TELLS IT TO PUT THE TABLE RIGHT HERE. 
\begin{center}
%TABULAR FORMAT IS THE WORD HERE; the c's represent centered
%columns, and the vertical bars represent vertical lines. 
%Lines are broken by \\, and columns are separated by the
%ampersand.
\begin{tabular}{|c|c|} \hline
Temperature & Voltage Drop \\
\hline
\hline
310K & 0.6761V$\pm$0.0004V\\
\hline
300K & 0.7064V$\pm$0.0005V\\
\hline
77K & 1.5318V$\pm$0.001V\\
\hline
\end{tabular}
\end{center}
\caption{Sample table}\label{normtable}
\end{table}

%The deluxtable always seems to generate a bunch of trivial errors. 
%Don't worry about them - even the AASTEX sample that comes with their
%official distribution generates a string of complaints.  Just check the
%dvi file and see if it looks okay.  (That is, of course, unless you
%plan to submit something to a real journal.)

%[AOWING TO VARIOUS PECULIARITIES, YOU SHOULD HAVE NO BLANK LINES INSIDE
%THE DELUXETABLE ENVIRONMENT--IN CONTRAST TO ALL OTHER PLACES IN TEX.

\begin{deluxetable}{crrcrrl} %the crrrl's set text alignment
\footnotesize
\tablecaption{Sample table} \label{2abs}
\tablewidth{430pt}
\tablehead{
\colhead{Source} & \colhead{$\ell$} & \colhead{$b$} &
\colhead{$\tau_{max}$} &
\colhead{$v_{LSR}$} & \colhead{FWHM} & \colhead{ref, note}
}
\startdata
0624-058 (3C161) & 215.4 & --8.0  &    0.67      &  12.0 &   4.5 &   1,a
\\
3C161            & 215.4 & --8.0  &   0.88       &   7.6 &   2.5 &   1,a
\\
3C161(OH)        & 215.4 & --8.0  &  0.013       &   8.6 &   1.2 &   3
\\
PKS0605-08       & 215.7 & --13.5 &
0.80$^b$         &  7.3  &   8.9 &   2
\\
0530+04 (4C04.18)& 200.0 & --15.3 & 0.8:         &  4.3: &   6.7:&   2
\\
3C135            & 200.5 & --21.0 & $\lesssim 0.11$&\nodata &\nodata & 2
\\
PKS0533-12       & 215.4 & --22.2 & 0.36         &  3.9  &   8.0 &    2
\\
\enddata
\tablerefs{(1) Mebold {\it et al.} (1981), Mebold
{\it et al.} (1982); (2) Crovisier, Kaz\`es, and Aubrey (1978);
(3) Dickey, Crovisier, and Kaz\`es (1981).}
\tablenotetext{a}{Mebold {\it et al.} (1982) list 3 components in
addition to the 4 listed here.}
\tablenotetext{b}{We have not listed a second, weaker Gaussian component
because of poor signal/noise.}
\tablecomments{This comment applies to the whole table and you can
put it either in front or behind the other comments.}
\end{deluxetable}

\end{document}

Note that anything here is skipped by the interpretor.  You can use it
as scratch paper or to keep templates handy when working on a document. 
Alternately, if you're trying to fix a document with a whole bunch of
bugs in it, temprarily tossing a \end{document} in the middle of the
file will save time, since the compiler won't bother trying to decipher
problems.

